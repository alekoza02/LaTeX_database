\section{2/10/24}

\subsection{[$\checkmark$] Ripasso eq. Schr\"{o}dinger}

From last time we can change the $(\vec{k}\vec{r} - Et)$ into $(\vec{p}\vec{r} - Et) / \hbar$ from de Broglie relations.

So what about particles that moves much slower than the speed of light and consequentely have mass?

$$ E = \frac{1}{2} mv^2$$
%
Considering that $\vec{v} = \frac{\vec{p}}{m}$ we can rewrite the previous formula as:

$$ E = \frac{(\vec{p})^2}{2m} $$
%
If we then consider that $E = \hbar \omega$ since $\vec{p} = \hbar \vec{k}$, we can write what is now called a "Matter Wave":

$$ \Psi_{p} (\vec{r}, t) = Ae^{i \left( \frac{\vec{p}\vec{x} - Et}{\hbar} \right)} $$
%
Where the terms of energy and momentum are the formulas described for the matter equations.

Il prof. spiega roba in maniera non estremamente chiara, per il ripassone, leggi il PDF di Uccirati.

\hrulefill
\begin{itemize}

    \item Linear operator $\longrightarrow$ is something that operates in a linear way towards a function (taking a derivative, or multiplying by a certain factor)
    \item Wave packet $\longrightarrow$ is a group of superposed waves which together form a travelling localized disturbance, especially one described by Schr\"{o}dinger's equation and regarded as representing a particle
    \item Probability density $\longrightarrow$ is the formula ($|\Psi(\vec{r},t)|^2$) that determines the probability to find a particle in a position $\vec{r}$ at the time $t$. In all the universe this equals to 1.
\end{itemize}

\subsubsection{Regole interessanti}

\begin{itemize}
    \item 2 waves which differ only by the phase, represent the SAME state ($|x|^2$)
    \item If we check that a wave is valid at a certain time, it will be valid for any time.
\end{itemize}

\subsection{[$\checkmark$] Course objectives}

\subsubsection{Complex vectors} 

Complex vectors can be:

\begin{itemize}
    \item vertical "KET" and symbolized with $\ket{v}$, and it's easy to immagine them as a column vector.
    \item horizontal "BRA" and symbolized with $\bra{v}$, and it's easy to immagine them as a row vector of the complex conjugate.
    \item the combination (BRAKET): $\braket{w|v}$ is a scalar product (dot).
\end{itemize}

\subsubsection{Dagger}

$\bra{v} = (\ket{v})^{\dagger}$ which is the simple term for "Hermitian Conjugation transpose \& compl. conjugation".
The $\sqrt{\braket{v|v}}$ equals to taking the norm, such as $\braket{\Psi|\Psi} = 1$

\subsubsection{Orhonormal basis}

Are a set of basis such as $\braket{i|j} = \delta_{ij}$ such as it is equal to 1 if $i = j$ or 0 if $i\neq j$. This means that the components $v_i$ is given by $v_i = \braket{i|v}$ 

\subsubsection{Hilbert space}

It is a complex vector space with a complex scalar product, with some rules:

\begin{itemize}
    \item States of quantum system are vectors in a Hilbert space with unit norm. $\ket{\Psi}$ and $\ket{\Psi}e^{i\phi}$ represent \underline{SAME} state.
    \item This means that $\hat{A}$ applied to $\ket{v}$ is equal to $\hat{A}\ket{v}$ and also $\hat{A}(a\ket{v} + b\ket{w}) = a\hat{A}\ket{v} + b\hat{A}\ket{w}$
    \item Note that $\hat{A}$ can be represented by a matrix. So an equivalent writing should be $\braket{i|\hat{A}|j} \equiv A_{ij}$
    \item $\hat{A}$ is hermitian if $\hat{A} = \hat{A}^{\dagger}$ and $(A^{\dagger})_{ij} = (A_{ji})^*$
\end{itemize}

\subsubsection{Hermitian operators properties}

\begin{itemize}
    \item Diagonalizable: $\hat{A} \ket{e_i} = \lambda_i \ket{e_i}$ where $\ket{e_i}$ is an eigenvector and $\lambda_i$ is an eigenvalues
    \item Can choose an orthon basis of eigenvalues.
    \item There's a proof that the eigenvalues are Real numbers.
\end{itemize}

\subsection{[X] Exercise}

$\ket{1}$ = (0 1)
$\ket{2}$ = (1 0)

\begin{itemize}
    \item Verify that $\ket{1}$ and $\ket{2}$ form an orthonormal basis
\end{itemize}

\hrulefill

$$\ket{1} = \begin{bmatrix} 1 \\ i \\ \end{bmatrix}  \frac{1}{\sqrt{2}}$$

$$\ket{2} = \begin{bmatrix} 1 \\ -i \\ \end{bmatrix} \frac{1}{\sqrt{2}}$$

\begin{itemize}
    \item Verify that $\ket{1}$ and $\ket{2}$ form an orthonormal basis
\end{itemize}

\hrulefill

\begin{itemize}
    \item Write $\ket{v} = \begin{bmatrix} 5+i \\ 7-i \\ \end{bmatrix}$ as $\ket{v} = v_1\ket{1} + v_2\ket{2}$ and $\ket{v} = v_1'\ket{1'} + v_2'\ket{2'}$
\end{itemize}

\hrulefill

\begin{itemize}
    \item Given this: $\hat{A}\ket{1} = \ket{2}$
    \item And this: $\hat{A}\ket{2} = \ket{1}$
    \item Write $\hat{A}$ as a matrix
\end{itemize}

