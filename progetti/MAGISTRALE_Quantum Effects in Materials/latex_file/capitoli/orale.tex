\section{13/11/24}

\section{[$\checkmark$] ORALE}

\begin{itemize}
    \item C1: Qualcosa sullo spin (Stern - Gerlac) Obtain the commutators and $\hbar^2 S(S+1)$ e il calcolo dei vari stati di singoletto, tripletto ++, --, +-, -+. Caso specifico $\ket{00}$. Sapere risultato di $J^2$ effetto su $\ket{l,l,+}$.  
    \item T1: Potential step con $E > V_0$ (matematica: solo soluzione prova e derivata seconda = $\Psi$). Calcola solo la trasmissione. 
    \item M1: Hartree Fock equations and state distribution. 
    \item C2: Scritto H come Gauge, Landau level $\rightarrow$ harmonic oscillator (il giappo scrive come un dannato).
    \item T2: Somma di (spin?) $a\ket{++} + b\ket{-+}$ e ladder operator
    \item M2: Bloch teorema \& overlap
    \item C3: Commutatori $[S_z, S_x] = i\hbar S_y$ usa matrici di Pauli. Stern-Gerlac. Uncertainty principle. Calcolo Normalization
    \item T3: Armoniche sferiche $R(r)sin(\theta) sin(\phi)$, calcolo dei numeri (sostituisce il seno con esponenziale complesso). Continua con esercizio sulla time evolution (dimostrazione completa). Discorso sul fatto che la fase non cambia lo stato del sistema.
    \item M3: Basis set e segno di $\prod$ e $FC = SCE$
    \item C4: (FRANCESCA) Perturbation theory $H = H_0 + \lambda H_1$, example of application in H atom. Zeeman effect - Hydrogen atom.
    \item T4: Commutatori commutano e Uncertainty principle (non si ricordava $\hbar / 2$ ma va tutto bene). Calcolo probabilità.
    \item M4: Charge density
    \item C5: $\alpha$ - decay
    \item T5: Time evolution
    \item M5: Hartree Fock equation $FC = SCE$, linearizzazione e basis set
    \item M6: Slater determinant, DFT working principle 
    \item M6: Full CR
\end{itemize}

28 28 30, 26, 23