\section{13/11/24}

\section{[$\checkmark$] MASCHIO 2.0}

\subsection{Operators \& Slater determinants}

Let's start from the $\hat{H}\Psi = E \Psi$, and then introduce the ORBITAL approximation, which is a mono electronic function of 1 electron.

$$\chi (\mathbf{x}) = \chi(\mathbf{r};\omega) = \psi(\mathbf{r}) \sigma(\omega)$$

We assume to have these SPINORBITALS, how do we handle them? What properties do they have to need? They needs to be ORTHONORMALS. Once we have them, we can separate them into a product of spinorbitals.

\subsection{Determinante di Slater}

Rappresenta una matrice, in cui ogna colonna rappresenta l'elettrone in uno degli spin-orbitali, mentre ogni colonna rappresenta uno scambio tra gli elettroni: ogni colonna rappresenta tutti i possibili spin-orbitali, ogni riga rappresenta ogni possibile elettrone.

è un modo per rappresentare tutte le possibile configurazioni degli elettroni in un modo comodo.

Il determinante è la funzione d'onda che rappresenta questo stato. Questo perchè il modo in cui calcoli il determinante, va a prendere tutti i contributi di un elettrone in tutti gli orbitali, che corrisponde a considerare la sovrapposizione dei vari orbitali.