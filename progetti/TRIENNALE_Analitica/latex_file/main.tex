\documentclass{article}
\usepackage{geometry}
\usepackage{amssymb}
\usepackage{amsmath}
\usepackage{xcolor}
\usepackage{booktabs}
\usepackage{graphicx}
\usepackage{float}
\usepackage{caption}
\usepackage{xcolor}
\usepackage[italian]{babel}
\usepackage{chemfig}

\setcounter{page}{0}

\geometry{
	left=.7in,  
	right=.7in, 
	top=.7in,   
	bottom=.7in,
}


\title{Analitica}
\author{Alessio Cimma}

\begin{document}

\maketitle

\begin{center}
	\includegraphics*[width=0.22\linewidth]{../images/logo.png}
\end{center}

\tableofcontents
\newpage

\section{ESAME}
\subsection{Formalismo}
Conteggi di "NON SIAMO AL BAR, MA IN UN'AULA UNIVERISTARIA": 8
\\\\
NON SI ECCITANO, ma si portano ad uno stato eccitato e ritornano al loro stato fondamentale rilasciando un quanto di energia (range: UV - Visibile).
\\\\
Le emissioni non sono DETERMINATE, ma sono CARATTERISTICHE
\\\\
Non si utilizza l'espressione più o meno
\\\\
Il campione non INTEREAGISCE ma si RIPARTISCE tra la fase stazionaria e mobile nella cromatografia
\subsection{Tempi}
Partenza: alle 9.10
Durate:
\begin{itemize}
	\item Number 1) 40 minuti
	\item Number 2) 50 minuti
	\item Number 3) 40 minuti
\end{itemize}
\subsection{Voti}
Candidato 1: 
\begin{itemize}
	\item Voto personale: non superato
	\item Voto finale: non superato
\end{itemize}
Candidato 2: 
\begin{itemize}
	\item Voto personale: non superato
	\item Voto finale: non superato
\end{itemize}
Candidato 3: 
\begin{itemize}
	\item Voto personale: 19-22
	\item Voto finale: 25
\end{itemize}
\newpage
\subsection{PRENESTI}
\subsubsection{Domanda n.1.1)}
Q: Era un metodo assoluto o relativo la taratura esterna? (Relativo) Mi parli invece del tipo di metodo assoluto più conosciuto e la differenza tra ass. e rel.
\\\\
A: Relativo. INCOMPLETA
\subsubsection{Domanda n.2.1)}
Q: Acido solforico e solfidrico, disegna formule e spiega cos'hanno in comune e cosa di diverso e come usarle in chimica analitica?
\\\\
A: $H_2SO_4$ e $H_2S$, sono entrambi bi-protici, con i relativi residui acidi ($SO_4$ e $S$).
\\\\ 
PRIMO:
\\- Si dissocia completamente in acqua
\\- Equilibri di dissociazione (stadio per stadio): $H_2SO_4 \rightarrow H^+ + HSO_4^-$ (anfolita = $HSO_4^-$) e successivamente $HSO_4^- \rightleftharpoons H^+ + SO_4^{2-}$
\\- Sottigliezza ma principale: è un equilibrio, quindi va in entrambe le direzioni, quindi si usano le $\rightleftharpoons$ nella seconda e una nella prima
\subsubsection{Domanda n.3.2)}
Q: Scriva un acido debole, il suo equilibrio di dissociazione e costante di dissociazione $K_a$ e come il solvente modifica questa costante?
\\\\
A: $K_a = [CH_3COO^-][H^+] / [CH_3COOH]$ partendo da $CH_3COOH \rightleftharpoons CH_3COO^- + H^+$, valori iù grandi o più piccoli significano che sarà più o meno forte. Se consideriamo il solvente, lo dobbiamo aggiungere ai reagenti $CH_3COOH + H_2O \rightleftharpoons CH_3COO^- + H_3O^+$ con i conseguenti cambiamenti nella formula di $K_a$
\subsubsection{Domanda n.3.3)}
Q: Caratteristica degli acidi che permette loro di avere $K_a$ invariato? 
\\\\
A: Essere forti al punto da avere effetto livellante. COMPLETA
\subsubsection{Domanda n.3.4)}
Q: Tecnica chimica che utilizza gli equilibri? La spieghi. Tipi di equilibri? 
\\\\
A: Volumetria, Redox, acido-base, solubilità e complessometrici. INCOMPLETA
\subsubsection{Domanda n.3.5)}
Q: In una titolazione in cui analizziamo un acido debole, usiamo un titolante forte da quale punto di vista?
\\\\
A: Una base forte, esempio ($KOH$ o $NaOH$). Si sfrutta il principio di neutralizzazione. INCOMPLETA
\subsubsection{Domanda n.3.6)}
Q: Indicatore in una titolazione acido-base?
\\\\
A: fenolftaleina, al suo punto di viraggio osserveremo un cambio di colore per esempio
\newpage
\subsection{PREVOT}
\subsubsection{Domanda n.1.1)}
Q: Avete condotto un xp in cui contavate il Ni nell'acciaio usando un std interno, dicci qualcosa sulla calibrazione, metodi visti e nel dettaglio lo std. interno e perchè dovremmo fare la calibrazione?
\\\\
A: Si baserà sul costruire una retta di taratura a diverse concentrazioni per ottenere l'andamento dei dati a varie concentrazioni (su un grafico x = conc. y = intensità del segnale (assorbanza, conteggi, intensità radiazione ($I_{\lambda}$)))
Nel caso del Ni nell'acciaio sulle y abbiamo usato l'ICP (quello col plasma) INCOMPLETA
\subsubsection{Domanda n.1.2)}
Q: Che tipo di tecnica è su cosa si basa, quale fenomeno fisico sfruttiamo?
\\\\
A: È una tecnica $\mathbf{atomica} \rightarrow$ il campione subisce un atomizzazione, ovvero alcuni atomi vengono altamente eccitati, che comincia ad emettere elettroni in un determinato range quando diventa plasma ovvero perdendo i suoi elettroni. Queste transizioni sono a determinate energie. COMPLETA
\subsubsection{Domanda n.1.3)}
Q: Cosa si intende per SPECIFICO e SELETTIVO nell'ambito delle specie chimiche? (2 risposte diverse a quanto pare)
\\\\
A: / INCOMPLETA
\subsubsection{Domanda n.1.4)}
Q: Che informazioni ci danno i range della retta di taratura?
\\\\
A: I range di validità e la linearità di un determinato comportamento (non si possono estrapolare i dati) COMPLETA
\subsubsection{Domanda n.1.5)}
Q: La retta di taratura dovrebbe passare per l'origine ma non succede sempre, ma se non passa per essa, è importante?
\\\\
A: Dipende se noi consideriamo nella nostra retta di taratura l'origine. INCOMPLETA
\\\\
Prenesti: E ALLORA LA LEVI, NON DISEGNI L'ORIGINE SE NON HA MESSO IL PUNTINO (un calmissimo Prenesti che giudica un grafico disegnato alla lavagna)
\subsubsection{Domanda n.1.6)}
Q: Come sfrutto la retta per risalire alla concentrazione (nello specifico di questo esperimento del Ni)? Come faccio a ricondurre i dati prelevati al loro vero valore?
\\\\
A: Utilizziamo il valore del coefficiente angolare, esprimendo la Conc. in funzione dell'intensità: da $I_{\lambda} = mC + b \rightarrow C = (I_{\lambda} - b)\/m$ INCOMPLETA
\subsubsection{Domanda n.1.7)}
Q: Se abbiamo un segnale fuori dall'intervallo di taratura, lo posso sfruttare? È lecito? E se ad un certo punto smettesse di essere lineare l'andamento?
\\\\
A: No, perchè è fuori dal range che abbiamo TARATO (non range lineare), più che altro è che non sappiamo se l'andamento rimane lineare superato questo valore. Nel caso smettesse di essere lineare, dobbiamo diluire il campione, questo perchè probabilmente esiste una limitazione della macchina, ecco perchè ci vogliamo ricondurre nel range di linearità. COMPLETA 
\subsubsection{Domanda n.1.8)}
Q: Perchè è saggio diluire poco piuttosto che tanto?
\\\\
A: Effetto matrice, o comunque problemi di tipo sistematico, mi discosto meno dalle condizioni iniziali. Ovvero, aumentando il numero di operazioni di 1, introduciamo un'errore sistematico. COMPLETA

\subsubsection{Domanda n.2.1)}
Q: Domanda generale sulle tecniche cromatografiche, caratteristiche, utilizzi e cosa accomuna tutte le cromatografie?
\\\\
A: 2 grandi famiglie, planare e verticale o in liquido e gas. L'obiettivo è quello di separare diverse sostanze presenti dentro al campione. Questo viene fatto in base alla loro velocità di propagazione e il loro tempo di ritenzione. In comune: presenza solvente, fase stazionaria (quello che intereagisce e rallenta la diffusione delle particelle) e mobile (eluente con il campione). Il principio alla base comune si basa su: velocità diverse, data dal RITARDO SELETTIVO, dovuto al fatto che i soluti intereagiscono in maniera differenziata, con un conseguente accumulo di ritardo. COMPLETA
\subsubsection{Domanda n.2.2)}
Q: Quali tipi di composti si prestano più alla cromatografia gassosa? Quali caratteristiche devono avere?
\\\\
A: Devono essere volatili, se è difficile renderlo gasssoso, non sarà facile farlo. Dunque per facilitarlo dobbiamo aumentare la T, ma allora non deve neanche degradarsi con la T. COMPLETA
\subsubsection{Domanda n.2.3)}
Q: Cos'è il tempo morto in cromatografia?
\\\\
A: È una caratteristica del sistema cromatografico, ed è il tempo che ci mette una sostanza che non intereagisce con la fase stazionaria, che può coincide con l'eluente, ad uscire dal cromatografo. COMPLETA
\subsubsection{Domanda n.3.1)}
Q: Mi parli della spettrofotometria e analisi qualitativa / quantitativa. Il grafico come lo leggo?
\\\\
A: /, qualitativo = tipo di molecola (scadente), quantitativa = quantità di materia / concentrazione (ottima) usande Lamberto Birra $A_{\lambda} = \varepsilon_{\lambda}bc$. INCOMPLETA 

\newpage

\section{MATERIALE}
\subsection{Elenco PREVOT}
\begin{itemize}
	\item Misurazioni chimiche $\checkmark$
	\item Campionamento e conservazione $\checkmark$
	\item Errore $\checkmark$
	\item Taratura - parametri metodo $\checkmark$
	\item Spettrofotometria molecolare UV-VIS $\checkmark$
	\item Introduzione cromatografia $\checkmark$
	\item Cromatografia liquida $\checkmark$
	\item Gascromatografia $\checkmark$
	\item Spettroscopie atomiche $\checkmark$
	\item Introduzione alla spettrometria di massa $\checkmark$
	\item Introduzione all'ICP-MS $\checkmark$
	\item FORS (Prof. Aceto)
	\item Analisi Cu
	\item Analisi Fenolo
	\item Analisi Ni
\end{itemize}
\subsection{Elenco PRENESTI}
\begin{itemize}
	\item Equilibrio chimico $\checkmark$
	\item Equilibrio acido - base $\checkmark$
	\item Equilibrio Redox $\checkmark$
	\item Equilibrio complessazione $\checkmark$
	\item Equilibrio di solubilità $\checkmark$
	\item Pretrattamenti $\checkmark$
	\item Analisi volumetrica $\checkmark$
	\item Determinazione pH $\checkmark$
\end{itemize}

\newpage

\section{APPUNTI}
\subsection{Misurazioni e campionamenti}
\begin{itemize}
	\item Se tecnica completamente non distruttiva in-situ $\rightarrow$ Campionamento
	\item Il campionamento è responsabile tra il 30-50\% dell'errore totale, la fase analitica solo il 5\%
	\item Non devono essere presenti BIAS di campionamento come per esempio il prelievo dalla superficie
	\item Dimensioni adeguate per contenere l'etereogenità del campione
\end{itemize}
%
Esiste un piano di campionamento in cui viene definito il corretto approccio per analizzare un campione. Principalmente deve essere prelevato e conservato in modo da non modificare le proprie caratteristiche chimico-fisiche. Esso comprende:
\begin{itemize}
	\item definizione obiettivo
	\item strategia di campionamento
	\item indicare la matrice da campionare
	\item metodo di campionamento
	\item numerosità campioni
	\item addetti al campionamento e skills richieste
	\item trasporto e stoccaggio
	\item controllo qualità
	\item sicurezza
	\item documentazione richiesta
\end{itemize}
%
Il campionamento può avere dei bias come per esempio:
\begin{itemize}
	\item Forma delle particelle: rotonde vs spigolose
	\item Adesività superficiale: sticky vs non sticky
	\item Moviemento differenziale verticale: effetto setaccio
	\item Cambiamento composizione: Variazione di acqua / comp. volatili, degrado del campione, reazioni con contenitore 
\end{itemize}
%
Campionamenti: gas, liquidi, solidi:
\begin{itemize}
	\item Gas: Aspirazione, Espansione, Spostamenti di liquidi
	\item Liquidi: Pompe. Semplice da omogenizzare, distinzioni sulla provenienza (sistemi aperti o chiusi, contenitori aperti o chiusi)
	\item Solidi: Campionatore almeno 3 volte più grande del particolato.
\end{itemize}
%
Suddivisione tra duro, compatto e non. Quest'ultimo si divide poi in movimento e statico.
\\\\
CONSERVAZIONE:
\begin{itemize}
	\item PE, PP, PTFE (Polietilene, Polipropilene, Teflon): Ottimo per tutto ciò che non è organico
	\item PE migliore rapporto qualità/prezzo, non adatto per metalli
	\item PP il più economico
	\item PTFE il più resistente e costoso
	\item Vetro: Lo attacca solo $HF$ concentrato, può rilasciare tracce in casi estremi (lavaggi aggressivi richiesti)
\end{itemize}

\newpage

\subsection{Errori}
Una suddivisione maggiore può essere vista come: GROSSOLANA, SISTEMATICA e CASUALE
\\\\
Di quelle Sistematiche e casuali abbiamo l'errore ASSOLUTO e RELATIVO, dove il primo è soltanto il $\Delta$, mentre il secondo è il $\Delta / x_{vero}$
\\\\
Le sistematiche si dividono poi in 4 suddivisioni:
\begin{itemize}
	\item bias: generico modo per descrivere un errore sistematico costante o proporzionale
	\item strumentali: inesattezza della calibrazione dello strumento utilizzato
	\item metodologici: comportamento non ideale di reattivi e reazioni
	\item personali: errori sistematici introdotti dall'operatore
\end{itemize}
%
Per prevenirli si può provare a:
\begin{itemize}
	\item Analizzare un campione standard
	\item Analizzare il campione tramite un metodo indipendente (provata affidabilità o riferimento)
	\item Analisi del bianco
	\item Analizzare campioni contenenti una diversa quantità (conosciuta) di analita
\end{itemize}

\newpage

\subsection{Taratura}
Insieme di operazioni che portano a stabilire una relazione tra i valori indicati dallo strumento e l'effettivo valore del misurando.
\\\\
2 Coefficienti di correlazione:
\begin{itemize}
	\item Pearson
	\item Spearman
\end{itemize}

\subsubsection{Pearson}
Assunzioni:
\begin{itemize}
	\item Entrambe le variabili sono continue
	\item I dati seguono una scala a intervalli o sono razionali
	\item Le variabili seguono una distribuzione normale
	\item La relazione tra le variabili è lineare
\end{itemize}
%
Cosa indica? Ci da un'informazione sul tipo di relazione tra i dati. Va da -1 a +1 e negli estremi indica una massima correlazione (diretta o indiretta tra i dati)

\subsubsection{Regressioni}
Esistono di varia natura, in generale però legano n° variabili indipendenti ad una variabile dipendente
Una tipologia è la retta ai minimi quadrati (LINEARE)

\subsubsection{Metodi di taratura}
\begin{itemize}
	\item Taratura esterna: si basa sulla curva di taratura ottenuta usando una concentrazioni note, ottenendo così il valore della risposta del sistema. Successivamente si ricava la concentrazione ignota esprimendola in funzione della calibrazione
	\item Taratura interna: Metodo degli standard interni.
\end{itemize}
%
Bisogna ricordare che nel caso di calibrazioni esterne non vale estrapolare i dati: puoi aggiungere nuovi valori nella retta di calibrazione, oppure diluire analita per avere risposta nel range corretto.
\\\\
Detto questo bisogna definire il range utile di linearità della risposta, ovvero nella LOQ (concentrazioni più bassa alla quale si possono fare misure quantitative) e la concentrazione più alta alla quale la curva devia dalla linearità.
\\\\
I vari standard (Primari e Certificati) devono possedere alcune proprietà:
\begin{itemize}
	\item Caratteristiche che non variazione temporale
	\item Facilimente purificabile
	\item Facilimente reperibile
	\item Essiccabile
	\item Impurezze note e costanti
	\item NO igroscopico
	\item Elevata massa molare
\end{itemize}

\newpage

\subsubsection{Metodi con i vari STANDARD}
Diversi metodi:

\paragraph{METODO DELLO STANDARD ESTERNO:} Usiamo uno standard esterno per tirar fuori una retta di taratura dello strumento.
\paragraph{METODO AGGIUNTE STANDARD:} Questo sistema ci permette di capire la concentrazione incognita di analita e se è presente un effetto matrice. Per bianco si intende una "riproduzione" del campione senza analita (matrice). Rispetto alla retta di taratura: Se parallelo -$>$ No effetto matrice. L'intercetta sarà la concentrazione dell'analita originale.
\subparagraph{Primario: } Soluzione di partenza dalla quale si preparano i secondari (creata da LAB, è un elenco di 15-20 elementi certificati, tutti gli altri sono secondari)
\subparagraph{Secondario: } Soluzione con una concentrazione variabile di standard primario. Si aggiungerà alle aliquote del campione originale e si valuterà l'andamento.
\paragraph{METODO DELLO STANDARD INTERNO:} Con questo sistema aggiungiamo un elemento alla soluzione di cui siamo sicuri che sia assente nel nostro campione. In questo modo possiamo ovviare ad errori come fluttuazioni del risultato dello strumento. Questo si può fare dividendo i segnali dell'analita a concentrazione ignota per la concentrazione dello std. interno, questo ci darà un nuovo valore con cui lavorare. Alla fine del processo ci basterà rimoltiplicare per la conc. dello std. interno per riottenere il vero valore della concentrazionedell'analita. 
\paragraph{LIMITE DI RILEVABILITÀ (LOD):} È definito come la minima concentrazione o quantità di una specie chimica in grado di fornire un segnale che può essere distinto con ragionevole fiducia da quello di un bianco. 

\newpage

\subsection{Spettrofotometria}
È basata sulle transizioni elettroniche molecolari. Ricade nell' UV-Visibile. Può essere considerata:
\begin{itemize}
	\item QUALITATIVA nel caso si riuscissero ad identificare precisi gruppi di cromofori. Questo funzionamento si basa sull'acquisizione in pratica di uno spettro caratteristico.
	\item QUANTITATIVA nel caso invece si andasse a confrontare l'intensità di ingresso e uscita del raggio (Tramite Lambert-Beer si ottiene la concentrazione)  
\end{itemize} 

\newpage

\subsection{Introduzione CROMATOGRAFIA}
Tecnica quali-quantitativa, si basa sulla distribuzione differenziale (data dal ritardo selettivo) delle diverse componenti fra le due fasi:
\begin{itemize}
	\item Fissa (stazionaria): Non per forza solida, può anche essere un gel o un liquido. Basta che non sia miscibile con la fase mobile
	\item Mobile (eluente): 
\end{itemize}
Ci si basa sull'introdurre in soluzione il campione o volatilizzarlo (in quantità minime). In base all'affinità delle componenti del campione, queste si tratterranno per più tempo nella fase fissa / mobile, modificando quindi la loro velocità di eluizione.
\\
La fase una volta inserita nella colonna, non rimmarrà sempre della stessa larghezza, ma comincerà ad allargarsi a causa di vari motivi:
\begin{itemize}
	\item Porosità e cammini leggermente diversi: \colorbox{yellow}{diffusione di Eddy}
	\item \colorbox{yellow}{Diffusione longitudinale}: rilevabile nella Gas-Cromatografia dato l'elevato coefficiente di diffusione
	\item trasferimento di massa in fase mobile: \colorbox{yellow}{Le zone adiacenti alle pareti comporteranno un'eluizione più lenta}
	\item trasferimento di massa in fase mobile stagnante: \colorbox{yellow}{I pori possono trattenere la fase mobile }
	\item trasferimento di massa in fase stazionaria: \colorbox{yellow}{Più è alto} il suo \colorbox{yellow}{coefficiente di diffusione}, e \colorbox{yellow}{più differenza di cammino ci sarà} tra quello che rimane intrappolato e quello che invece riesce a passare 
\end{itemize}

\paragraph{Meccanismi di separazione: }Esistono vari sistemi per separare la fase mobile:
\begin{itemize}
	\item \colorbox{yellow}{Adsorbimento}: siti attivi sulla superficie della fase stazionaria che adsorbono la fase mobile
	\item \colorbox{yellow}{Ripartizione}: Fase stazionaria $\pm$ polare della fase mobile (normale se stazionaria più polare | inversa se eluente più polare), la differenza di velocità dipenderà dalla polarità della sostanza, e quindi a cosa si attaccherà di più
	\item \colorbox{yellow}{Scambio ionico}: sulla fase fissa saranno presenti macromolecole che cederanno cationi o anioni e solo alcune molecole trasportate dalla fase mobile perderanno tempo a legarsi a questi ioni.
	\item \colorbox{yellow}{Esclusione dimensionale}: saranno presenti pori di dimensioni controllate che tratterranno particelle in base alla loro dimensione
	\item Affinità: 
	\begin{itemize}
		\item ANALISI QUANTITATIVA: Esiste una relazione lineare tra concentrazione analita e area del picco cromatografico
		\item ANALISI QUALITATIVA: Si può fare un'attribuzione basata sulla coincidenza dei tempi di ritenzione standard di singole sostanze pure 
	\end{itemize}
\end{itemize}

\subsection{Cromatografia liquida}

Importante che non ci sia aria all'interno del campione, per evitare rallentamenti. Questo comporta la necessità di una buona pompa in ingresso con valvola e alta pressione.
\\I supporti possono essere di vario tipo come a particelle e monolitico (polimero).
\\Può essere presente una precolonna il cui scopo è quello di proteggere la colonna principale. Può essere facilmente sostituita.
\\In lab abbiamo usato il C18 come fase non polare (per la ripartizione)
\\Il rilevatore deve avere buone caratteristiche di range e velocità di analisi.
\\Di solito si tratta di uno spettrofotometro (singolo, variabile, array) oppure di un spettrofluorimetro, indice di rifrazione, elettrochimico o conducimetrico 
\\Scelta della tecnica:
\begin{itemize}
	\item Peso molecolare
	\item Solubile in acqua o solvente organico
	\item Scelta della colonna
	\item Controllo fattore di selettività e Capacità
	\item Scelta numero piatti teorici
	\item Appiattire eventuali gradienti
\end{itemize}
\paragraph{\colorbox{yellow}{Derivatizzazione:} } Si basa sull'attaccare un determinato gruppo ad una specie per aumentare la sua visibilità al rilevatore.
Si può fare sia pre che post colonna, i vantaggi e svantaggi sono la velocità di preparazione del campione e il costo dell'apparecchiatura da usare.

\subsection{Cromatografia gassosa}

Impiegata per la separazione di sostanze volatili. La fase mobile sarà composta da un gas inerte + campione.
\\- GAS-LIQ: Meccanismo di ripartizione: il liquido è legato covalentemente al solido inerte di supporto
\\- GAS-SOL: Meccanismo di adsorbimento.
\\\\
Distinzione tra le colonne "impaccate" e "capillari". Il vantaggio della prima sta nella riduzione della diffusione di Eddy. La seconda invece permette una diffusione più semplice, riducendo il tempo di analisi. \colorbox{yellow}{La temperatura si} \colorbox{yellow}{sceglie come media dei punti di ebollizione dei vari composti.}

\paragraph{Derivatizzazione: } Aumentare volatilità degli analiti e la sensibilità della rivelazione

\newpage

\subsection{Spettroscopia}
Modalità di funzionamento (spettro dovrebbero essere linee discrete, ma non lo sono a causa di interferenze $\rightarrow$ LARGHEZZA DI RIGA EFFICACE (FWHM)):
\begin{itemize}
	\item Assorbimento: Solo certi atomi possono assorbire la radiazione monocromatica
	\item Emissione: Gli elementi presenti sottoforma di vapore atomico emettono un surplus di energia come radiazione luminosa
	\item Fluorescenza: /
	\item Spettrometria di massa: /
\end{itemize}
Modalità di atomizzazione (nebulizzazione del campione):
\begin{itemize}
	\item Fiamma (FAAS): Getto di gas nebulizza il liquido, che viene ulteriormente spezzato da una fiamma. Questo avviene in mezzo a monocromatore e rilevatore.
	\item Termoelettrico (ETAAS): Stessa cosa di prima, il calore e fornita da corrente ad alta potenza. (misura sempre effettuata in assorbanza)
	\item Plasma di $Ar$
	\item Arco elettrico
\end{itemize}

\paragraph{Funzionamento: } Assorbanza seguendo la legge di Lambert-Beer. Gli atomi possono assorbire frequenze ben precise. Ecco perchè affinchè l'emissione sia affidabile, la fonte dev'essere il più monocromatica possibile. Si usa una lampada a catodo cavo, e il suo funzionamento è definito monoelementare perchè tramite lo sputtering emette singoli atomi dal catodo che per effetto di urti con gas di riempiemnto emettono il loro spettro di emissione. Se lo stesso elemento è presente nell'analita, verrà identificato.
\paragraph{Interferenze: } Possono essere spettrali se una specie interferente emette o assorbe molto vicino all'analita mentre è un'interferenza chimica se l'analita in qualche modo reagisce o forma composti, modificando il suo spettro di emissione.
\\\\
Si può aumentare la sensibilità quando si analizzano certi elementi usando la tecnica a vapori freddi (metalli pesanti / di destra)

\paragraph{Emissione: } Quello che alla fine ci interessa è quantificare l'intensità di emissione. In fiamma siamo tra i 2000-3000°, mentre al plasma siamo attorno ai 6000-7000°. In particolare l'ICP si basa su una torcia ICP che è costituita da una bobina che genera un campo magnetico talmente potente da ionizzare l'argon presente, il quale a sua volta intereagendo con il campione, lo "desolve", "vaporizza" e "dissocia". Una volta eccitato emette luce, raccolta dallo spettrofotometro.
\\
\paragraph{LIBS: } Laser Induced Breakdown Spectroscopy usa pulsazione laser come sorgente di eccitazione. È una tecnica qualitativa e superficiale. È microdistruttiva.
La particolarità è che viene usata l'emissione dei fotoni che passano dalle shell più esterne a quelle più interne. Questo perchè il campione è stato sublimato e ionizzato; nel processo di riacquisizione degli ioni uno spettro caratteristico viene riemesso.

\subsection{Equilibrio chimico}
\paragraph{Equilibrio dal punto di vista cinetico:} Anche se abbiamo costanti di velocità diverse, non significa non raggiungiamo un equilibrio: sarà semplicemente più spostato in una certa direzione. 
\paragraph{Equilibrio dal punto di vista termodinamico:} Tendono tutte ad uno stato di MAGGIORE ENTROPIA e MINORE ENTALPIA, e saranno favorite dal freddo se esotermiche e dal caldo se endotermiche. In questo modo possiamo ottenere $\Delta G = \Delta H - T\Delta S$ dove se è minore di 0 allora spontaneo

\paragraph{Quoziente di reazione:} È un coefficiente che si calcola come la costante di equilibrio, ma si calcola quando la reazione non è ancora terminata, la differenza positiva o negativa rispetto a $K$ ci dirà in quale senso procede la reazione 

\newpage
\subsection{RIPASSONE: }

\begin{itemize}
	\item \colorbox{yellow}{Ione ammonio}: $NH_4^+$, Ione ammide: $NH_2^-$
	\item \colorbox{yellow}{Ione idrossonio}: $H_3O^+$, Ione idrossido: $OH^-$
\end{itemize}

\paragraph{NOMENCLATURA: } 
\begin{itemize}
	\item Composti binari: 
	\begin{itemize}
		\item Ossidi $\rightarrow$ metallo, ossigeno $\rightarrow$ Ossido metall - OSO / ICO
		\item Anidridi $\rightarrow$ non metallo, ossigeno $\rightarrow$ Anidride (IPO / PER) non metall - OSA / ICA
		\item \colorbox{yellow}{Idruri} $\rightarrow$ metallo, idrogeno $\rightarrow$ Idruro metall - OSO / ICO
		\item Idracidi $\rightarrow$ idrogeno, alogeno $\rightarrow$ Acido alogen - IDRICO
		\item Sali binari $\rightarrow$ metallo, non metallo $\rightarrow$ Cloruro di Sodio
	\end{itemize}
	\item Composti ternari: 
	\begin{itemize}
		\item \colorbox{yellow}{Idrossidi} $\rightarrow$ metallo, ossigeno, idrogeno $\rightarrow$ Idrossido metall - OSO / ICO
		\item \colorbox{yellow}{Ossiacidi} $\rightarrow$ idrogeno, non metallo, ossigeno $\rightarrow$ acido (IPO / PER) non metall - OSO / ICO [ottenute da Anidride + $H_2O$, nel caso di $B$, $P$ e $Si$ sono 2 le molecole di $H_2O$]
		\item Sali ternari $\rightarrow$ metallo, non metallo, ossigeno $\rightarrow$ Non metallo - ITO / ATO di metall - OSO / ICO 
	\end{itemize}
	\item Composti quaternari: 
	\begin{itemize}
		\item Sali quaternari $\rightarrow$ metallo, idrogeno, non metallo, ossigeno $\rightarrow$ /
	\end{itemize}
	\item Ioni positivi (cationi): 
	\begin{itemize}
		\item Monoatomici: Ione metall - OSO / ICO
		\item \colorbox{yellow}{Poliatomici}: Ione elemento - ONIO [eccezione $\rightarrow$ $NH_4$ = Ammonio]
	\end{itemize}
	\item Ioni negativi (anioni): 
	\begin{itemize}
		\item Monoatomici: Ione elemento - URO
		\item 1) \colorbox{yellow}{Poliatomici}: Ione elemento - URO [eccezione $\rightarrow$ $OH^-$ e $O_2^{2-}$ = Idrossido e perossido]
		\item 2) \colorbox{yellow}{Poliatomici}: Ione (di-tri-tetra) elemento - ATO [eccezione $\rightarrow$ $OH^-$ e $O_2^{2-}$ = Idrossido e perossido]
		\item 2) Ossoanioni: Ione (di-tri-tetra) elemento - (ITO / ATO) [eccezione $\rightarrow$ $OH^-$ e $O_2^{2-}$ = Idrossido e perossido]
	\end{itemize}
\end{itemize}

\paragraph{ACIDI: } 
\begin{itemize}
	\item \colorbox{yellow}{Arrhenius}: Acido se rilascia $H^+$
	\item \colorbox{yellow}{Bronsted-Lowry}: Acido se dona $H^+$
	\item \colorbox{yellow}{Lewis}: Acido se può accettare doppietto elettronico
	\item Poliprotici: 2 step di dissociazione
\end{itemize} 

Solventi: Se sono polari avranno $\varepsilon$ alta e un momento di polo.
\\\\Nelle RedOx il modo in cui sono legati è: $\text{sign}(\Delta G^\circ) = -\text{sign}(\Delta E^\circ)$

% -------------------------------------------------------------------------------------------------------
\newpage

\section{DEFINIZIONI}

\subsection{RedOx / Roba di Prenesti}
\colorbox{yellow}{Ossidazione:} perdita elettroni\\
\colorbox{yellow}{Riduzione:} acquisto elettroni

\paragraph{Reazioni di dismutazioni:} Sono reazioni in cui lo stesso elemento chimico si ossida e si riduce. Questo elemento deve avere almeno tre stati di ossidazione stabili.
\paragraph{Elettrochimica:} \begin{itemize}
	\item Fenomeno elettrolitico: quando l'elettricità fa avvenire una reazione chimica (avviene nelle celle elettrolitiche) $\rightarrow$ Leggi di Faraday
	\item Fenomeno galvanico: quando una reazione chimica fa avvenire elettricità (avviene nelle celle galvaniche) $\rightarrow$ Leggi di Volta e Eq. di Nernst
\end{itemize}

\paragraph{Leggi di Volta: } \begin{itemize}
	\item \colorbox{yellow}{Prima legge di Volta}: Due metalli in contatto alla stessa temperatura generano una differenza di potenziale
	\item \colorbox{yellow}{Seconda legge di Volta}: In una catena, la prima legge si applica come se primo e ultimo fossero in diretto contatto
	\item \colorbox{yellow}{Terza legge di Volta}: Se il conduttore elettrolitico, la differenza di potenziale tra i vari metalli sarà diversa dal contatto diretto
\end{itemize}
\paragraph{Leggi di Faraday: } \begin{itemize}
	\item \colorbox{yellow}{Prima legge di Faraday}: La massa depositata su un elettrodo è proporzionale alla corrente passata
	\item \colorbox{yellow}{Seconda legge di Faraday}: Le masse di diversi elementi depositate saranno proporzionali (a corrente costante) al rapporto tra massa atomica e valenza dell'elemento
\end{itemize}

\paragraph{Elettrodi: }\begin{itemize}
	\item \colorbox{yellow}{Prima specie}: Metallo immerso in soluzione con ioni dello stesso metallo
	\item \colorbox{yellow}{Seconda specie}: Metallo ricoperto di sale poco solubile immerso in soluzione con anioni del sale 
	\item \colorbox{yellow}{Terza specie}: Metallo inerte con sciolte in soluzioni entrambe le forme ossidate e ridotte di un altro elemento
	\item \colorbox{yellow}{Quarta specie}: simili alla prima specie, ma sono gassosi e composti da $Pt$ + $H_2$
\end{itemize}

\paragraph{Spontaneità: } Sarà data $\Delta E^{\circ} > 0$, questo perchè $\Delta G^\circ = -nF\Delta E^\circ < 0$. La $E^\circ$ della cella verrà data da $E^\circ_{rid}-E^\circ_{ox} = E^\circ_{catodo}-E^\circ_{anodo} $

\paragraph{Hard - Soft: } \colorbox{yellow}{Classificazione in base alla polarizzabilità degli orbitali.} Hard = orbitali piccoli e difficili da ionizzare, soft il contrario. Hard-Hard, Soft-Soft. Hard ha un carattere più elettrostatico, Soft ha più un carattere covalente.

\paragraph{Chelati: } Sono più stabili dello stesso numero di monodentati attaccati, usate in analitica, perchè sono in grado di combinarsi 1:1 con lo ione metallico, rimuovendo specie dalla soluzione in maniera completa.
\paragraph{Effetti macrociclo: } Sono più stabili della catena aperta, resistono bene all'attacco di acidi forti, sono molto selettivi (alcuni rendono poi difficile la rimozione del metallo)
\paragraph{EDTA: } Lievemente acido o base in base all de/protonazione, molto utile in chimica analitica perchè avendo molti siti d'attacco, questi si attivano solo a certi pH, facendoci quindi capire la frazione di EDTA di forma X (intero) presente 

\begin{center}
	\includegraphics*[width=0.9\linewidth]{../images/EDTA.png}
\end{center}

\subsubsection{Solubilità}Principalmente sono 2 fattori competitivi che si scontrano: Energia reticolare e Energia di idratazione. Se quella reticolare è più grande sarà insolubile.
\paragraph{Elettrolita:} Sostanza che sciogliendosi in acqua forma ioni. La reazione che genera ioni si chiama DISSOCIAZIONI. Esistono i non elettroliti (zuccheri e alcol)
\paragraph{\colorbox{yellow}{Solubilità e Elettrolita:}} Attenzione alla differenza, il fatto che qualcosa sia poco solubile, non significa che non si dissoci completamente: solubilità indica la \% di campione che si scioglie, l'attributo elettrolita invece si riferisce alla \% di dissociazione del campione che si scioglie. Strani esempi:
\begin{itemize}
	\item $AgCl$ poco idrosolubile, ma dissociazione completa 
	\item $CH_3COOH$ molto idrosolubile, ma bassa dissociazione
\end{itemize}
Questa viene indicata come $K_{ps}$ (prodotto di solubilità) ed è calcolata come il prodotto di concentrazione degli ioni dissociati: $K_{ps} = K_{eq} [BaSO_4] = [Ba^{++}][SO_4^{--}] = S \cdot S = S^2$ e nel caso di stechiometria 1:1 abbiamo che le due S sono uguali tra loro.
In generale si può vedere la solubilità come la massima concentrazione possibile di un sale poco solubile ad una certa T.

Durante un processo, può essere usato Q per capire lo stadio di reazione $\rightarrow$ Q $< K_{ps}$ allora è insatura

\paragraph{IMPORTANTE:} 
Composti idrosolubili:
\begin{itemize}
	\item Cloruri, Bromuri, Ioduri di tutti i cationi (tranne $Ag^+, Hg^+, Pb^{2+}$)
	\item Sostanze con il catione ammonio $NH_4^+$
	\item Sostanze con gli anioni acetate e perclorato (tranne $KClO_4$)
	\item Quasi tutti i nitrati
	\item I solfati degli elementi del gruppo IA, di $Mg^{2+}$ e di $Fe^{2+}$
\end{itemize}
Composti scarsamente idrosolubili:
\begin{itemize}
	\item Solfuri
	\item Carbonati
	\item Fosfati
	\item Ossalati
	\item Solfati di ($Ca^{2+}$, $Sr^{2+}$, $Ba^{2+}$, $Pb^{2+}$, $Hg^{2+}$, $Ag^{+}$)
\end{itemize}

\paragraph{Fattori d'influenza per la solubilità:} 
\begin{itemize}
	\item Temperatura: questi processi sono endotermici
	\item Ione comune: Più uno ione è presente in soluzione, e più è difficile la solubilità
	\item pH: In base alla possibile reazione con $H_3O^+$ o $OH^-$, se una componente legandosi sposta l'equilibrio. 
	\\\colorbox{yellow}{La solubilità dei sali contenenti l'anione di un acido debole aumenta al diminuire del pH}
	\item Salinità: \colorbox{yellow}{Aggiungendo ioni NON comuni}, andremo a modificare l'equilibrio, aumentando il prodotto molare delgi ioni di sale.
	\item Complessazione: \colorbox{yellow}{Creando complessi, questi si tolgono dall'equazione}, permettendo a più sale di sciogliersi
\end{itemize}

\paragraph{Gravimetria:} Calcolo della massa o della variazione della massa

\subsection{Roba Prevot}
\paragraph{Principio di Le Chatelier:} Ogni azione esterna che modifica un sistema chimico, provocherà una reazione contraria da parte del sistema che tenderà ad annullarla

\paragraph{Acidi:} Dati $AH + B \rightleftharpoons A^- + HB$
\begin{itemize}
	\item \colorbox{yellow}{$AH$ è un acido}
	\item \colorbox{yellow}{$B$ è una base}
	\item \colorbox{yellow}{$A^-$ è la base coniugata  }
	\item \colorbox{yellow}{$HB$ è un acido coniugata}
\end{itemize}
Se un acido è forte, la sua base coniugata sarà debole e viceversa. Questa forza dipenderà dal solvente
\\
Un elenco dei solventi:
\begin{itemize}
	\item Protofili: Acquisiscono protoni [$NH_3$]
	\item Protogeni: Cedono protoni [$CH_3COOH, H_2SO_4,HCOOH$]
	\item Aprotici: non scambiano protoni a condizioni normali [$CH_3CN$, benzene (meglio toluene)]
	\item Anfiprotici: scambiano acidi e basi [$H_2O, CH_3OH$]
\end{itemize}

\paragraph{Effetto livellante:} Fenomeno per il quale è impossibile distinguere la forza di acidi o basi forti. Più base e più facilitano gli acidi, mandandoli in completa dissociazione, più sono acidi invece e più li ostacolano, facendo effettivamente scoprire qual è il più forte.

\hrulefill

\begin{itemize}
	\item Campione: soluzione da analizzare (pezzo di una lega)
	\item Campione rappresentativo: rispecchia le proprietà dell'insieme di estrazione
	\item Campione selettivo: non rispecchia le proprietà dell'insieme di estrazione
	\item Analita: sostanza da analizzare (cotenuto di un elemente nel pezzo di lega)
	\item Matrice: tutte le sostanze presenti nel campione che non siano l'analita
\end{itemize}
\hrulefill
\begin{itemize}
	\item Metodi chimici: reazioni chimiche (equilibri) $\rightarrow$ utilizzano vetreria
	\item Metodi chimici-strumentali: basato sempre su reazioni chimiche ma servono strumentazioni
	\item Metodi strumentali: non si basano su reazioni chimiche, serve solo uno strumento
\end{itemize}
\hrulefill
\begin{itemize}
	\item ANALISI quantitativa $\rightarrow$ quanto ce n'è?
	\item ANALISI qualitativa $\rightarrow$ cosa c'è?
\end{itemize}
%	
Può essere definita \colorbox{yellow}{ELEMENTARE se restituisce l'insieme di elementi presenti} e non di sostanze composte o specie in cui l'elemento si può trovare.
\\\\
Le analisi quantitative si suddividono in:
\begin{itemize}
	\item Metodo ASSOLUTO
	\item Metodo COMPARATIVO
\end{itemize} 
\hrulefill
\\\\Varie tipologie di errori:
\begin{itemize}
	\item Casuale $\rightarrow$ discostamento da un valore medio ottenuto con infinite misurazioni in condizioni di ripetibilità
	\item Sistematico $\rightarrow$ discostamento della media di un numero infinito di misurazioni da un valore vero 
	\item Di misura $\rightarrow$ discostamento di una misurazione da un valore vero
	\item \colorbox{yellow}{Precisione} $\rightarrow$ grado di concordanza tra risultati indipendenti ottenuti con condizioni stabilite
	\item \colorbox{yellow}{Esattezza} $\rightarrow$ grado di concordanza tra il valore medio da un grande insieme di prove e un valore di rif. accettato
	\item \colorbox{yellow}{Accuratezza} $\rightarrow$ grado di concordanza tra una misurazione e il vero valore del misurando 
	\item Scarto sperimentale $\rightarrow$ deviazione standard $(\sigma_{std})$
	\item Scostamento sistematico $\rightarrow$ Differenza tra media e valore di riferimento accettato
	\item Incertezza $\rightarrow$ Parametro che caratterizza la dispersione dei dati attribuiti ad un determinato misurando 
\end{itemize}
%
Taratura:
\begin{itemize}
	\item Correlazione: relazione tra due variabili
	\item Regressione: tipo e forma della relazione
\end{itemize}
%
Differenza tra SELETTIVITÀ e SPECIFICITÀ:
\begin{itemize}
	\item SELETTIVITÀ: \colorbox{yellow}{capacità di un metodo di non risentire di interferenze} causate da altre specie chimiche presenti. Più è selettivo e meno interferenze subisce
	\item SPECIFICITÀ: il non plus-ultra della selettività
\end{itemize}
%
Diversi tipi di standard:
\begin{itemize}
	\item Certificati: certificati da diverse agenzie, conterrà già l'analita all'interno della matrice dove verrà analizzato.
	\item Primari: Standard certificati da laboratori determinati SENZA alcun altro passaggio. Se subiscono anche solo un passaggio di reazione: secondari.
	\item Secondari: diluizioni dello standard primario
\end{itemize}
%
Terminologia degli strumenti:
\begin{itemize}
	\item Segnale: somma delle risposte dell'analita, fondo e interferenze (fondo e interferenze stessa varianza assunta)
	\item Fondo: componente strumentale che si misura in assenza di analita. Viene valutato con la misura del bianco.
	\item Rilevabile: se il picco dista più di 3 $\sigma_{std}$ dal picco del bianco (LOD) -$>$ analisi qualitativa
	\item Quantitativo: se il picco dista più di 10 $\sigma_{std}$ dal picco del bianco (LOQ) -$>$ analisi quantitativa
	\item Sensibilità: Capacità di distinguere tra bianco e analita. Corrisponde al coefficiente angolare della retta di taratura. C'è una relazione tra Sensibilità e LOD 
\end{itemize}
\hrulefill
\paragraph{Cromatografia: } Ci sono diversi termini da ricordare bene.
\begin{itemize}
	\item \colorbox{yellow}{Tempo morto: }tempo necessario affinchè la prima sostanza (di solito la componente della fase mobile) raggiunga il rilevatore
	\item \colorbox{yellow}{Tempo di ritenzione:} tempo necessario affinchè l'analita che ci interessa raggiunga il rilevatore (non viene considerato il tempo morto)
	\item Volume di ritenzione: volume minimo necessario affinchè l'analita riesca ad attraversare la colonna
	\item Volume morto della colonna: volume di spazio libero all'interno della colonna (interstizi)
	\item \colorbox{yellow}{Fattore di capacità $K$:} $K=n_f/n_m$ dove $n_f$ è il numero di moli della fase fissa e $n_m$ è il numero di moli della fase mobile -$>$ il rapporto indica quanto volume potenzialmente è contenibile all'interno della colonna.
	\item \colorbox{yellow}{K in funzione dei tempi:} $K = (t_r-t_m) / t_m$. È richiesto che sia compreso tra 1 e 10-15
	\item \colorbox{yellow}{selettività:} Espressa come il rapporto delle capacità: $\alpha = k_A / k_B$ è la capacità del sistema di eluire fasi diverse con velocità diverse
	\item \colorbox{yellow}{Efficienza:} Capacità di fornire picchi stretti. Si può descrivere in termini di $W_b$ e $N$ (Numero piatti teorici)
	\item \colorbox{yellow}{N: Piatto teorico} $\rightarrow$ Immaginatelo come un dischetto standard in grado di assorbire tot materia. Più ne hai e più tutto si muove lentamente e in maniera omogenea, aumentando l'efficienza. Si calcola come: $N = L/H =L^2/\sigma_{std}^2=16L^2/w_b^2$, questo valore dipende da un sacco di parametri come per esempio: diametro particelle, flusso, temperatura, viscosità solvente, dimensioni analita, impaccamento colonna 
	\item \colorbox{yellow}{Risoluzione:} Infine abbiamo la risoluzione, che è l'unione di più concetti, infatti si calcola come: 
	\\$R_S = (t_{RA} - t_{RB}) / 2(\sigma_A - \sigma_B)$
\end{itemize}

Struttura ICP-AES
\begin{itemize}
	\item Torcia ICP
	\item Nebulizzatore
	\item Camera di nebulizzazione
	\item Spettrometro
	\item Rilevatore
\end{itemize}

\hrulefill

\begin{itemize}
	\item \colorbox{yellow}{Analisi chimica:} Non si deve perdere l'analita.
	\item \colorbox{yellow}{Analisi di speciazione:} Non di deve perdere ne l'analita ne le forme chimiche in cui l'analita  è presente.
\end{itemize}

\begin{itemize}
	\item Reazioni irreversibili $\rightarrow$ COMPLETE
	\item Reazioni reversibili $\rightarrow$ INCOMPLETE
\end{itemize}

\newpage

%----------------------------------------------------------------------------------

\section{Pretrattamenti}

Operazioni FISICHE che si applicano al campione prima dell'analisi specifica. Consiste in vari passaggi come:

\begin{itemize}
	\item Essiccamento: eliminazione di acqua, dato che durante la conservazione potrebbe facilmente cambiare
	\item Macinazione: rende più omogeneo il campione (sottocampioni più rappresentativi). Facilmente attaccabili. Creazione di pastiglie.
	\item Setacciatura: eseguita per i solidi 
	\item Filtrazione: eseguita per i liquidi
\end{itemize}

\paragraph{Tecniche di macinazione:} Mortaio (diamante o agata) e pastello. Precauzioni: surriscaldamento locale del campione e contaminazioni da campioni precedenti.
\paragraph{\underline{Approcci all'analisi chimica}:} 

\begin{itemize}
	\item Analisi diretta sul campione
	\item Analisi per via umida
	\item Analisi in fase eterogenea
\end{itemize}

\paragraph{Pretrattamento del campione:} L'insieme di procedure per avere gli analiti in forma determinabile. Spesso è richiesta la lavorazione in soluzione. Queste tecniche comprendono:

\begin{itemize}
	\item \underline{Digestione umida}
	\item \underline{Fusione}
	\item Separazione con membrane
	\item \underline{Estrazione}
	\item Fotolisi ossidativa
\end{itemize}

\hrulefill

\subsection{Digestione UMIDA}

- Adatto per: determinazione di sostanze inorganiche (ioni metallici)
\\
- NON adatto per: sostanze organiche
\\\\
Caratteristiche richieste:
\begin{itemize}
	\item Capace di sciogliere completamente il campione
	\item Ragionevolmente veloce
	\item Se utilizza reagenti aggressivi, non devono interferire in futuro
	\item Reagenti con elevato grado di purezza
	\item Perdite per \underline{volatilità} insignificanti
	\item Né reagente né campione devono attaccare il contenitore
	\item Procedura sciura 
\end{itemize}

\paragraph{Dissoluzione in acqua:} Possibile in pochi casi. Funziona bene con sali ed elettroliti. Viene usata l'acqua ultrapura.
\paragraph{Dissoluzione in acido non ossidante:} 

\begin{itemize}
	\item \colorbox{yellow}{{$\mathbf{HCl}$}:} Usata per determinare specie inorganiche. Si può usare con i metalli, poichè la maggior parte di loro avrà un $E^{\circ}$ più negativo di $H^+ / H_2$. Può avvenire a freddo e a caldo. Gli acidi non ossidanti sono quelli che non hanno Ossigeni (per esempio $HCl$). Sono inoltre idrosolubili tutti cloruri (tranne quelli con: $Hg, Ag, Tl$)
	\item \colorbox{yellow}{{$\mathbf{HF}$}:} Usato con le matrici silicee, trasformandole in tetrafloruro di Si (aeriforme). Se è concentrato può generare acido esafluorosilicico ($H_2SiF_6$)
	\item \colorbox{yellow}{{$\mathbf{H_3PO_4}$}:} Poco usato perchè molti fosfati metallici sono insolubili e lo ione fosfato interferisce in analisi successive.
\end{itemize}

\paragraph{Dissoluzione in acido ossidante:} 
\begin{itemize}
	\item \colorbox{yellow}{{$\mathbf{HNO_3}$}:} È un forte ossidante a causa del suo ione nitrato, che una volta sciolto tende a generare $NO_2$ e $NO$. Ossida tutti i metalli tranne nobili e passivi. Se il metallo è refrattario, si ossidano ad alte T. Questi attacchi generano una patina sopra il campione, impedendo ulteriori attacchi. Quasi tutti i nitrati sono idrosolubili. L'anione nitrato è un complessante molto debole. Pecca dell'acido nitrico: alcuni ossidi precipitano quando si formano, quindi potrebbe convenire usare $HCl$ per generare dei cloruri e liberare $H_2$ 
	\item \colorbox{yellow}{{$\mathbf{H_2SO_4}$}:} Una caratteristica importante è che solfati metallici (prodotto di unione dell'acido con ione metallico) sono poco volatili. Ecco perchè è usato come agente essiccante. (non funziona su $Ca, Sr, Ba, Pb, Ag$)
	\item \colorbox{yellow}{{$\mathbf{HClO_4}$}:} Viene usato come colpo di grazia su ioni trattati con $HNO_3$, questo perchè è estremamente potente, ma molto instabile allo stesso tempo.
\end{itemize}

\paragraph{Dissoluzione in miscele di acidi:} 
Questi permettono di combinare le proprietà di diversi acidi (es. Ossidanti e Complessanti) o al contrario di moderare proprietà eccessive di altri. Si scioglie con un acido e poi lo si rimuove (se reca danni) con un altro (es. $HNO_4 + HClO_4$).
Un esempio di miscele di acido complessante + acido ossidante:
\begin{itemize}
	\item $HF + HNO_3$
	\item $HF + HClO_4$
	\item $HF + H_2SO_4$
\end{itemize}
%
Ora più precisamente, vediamo le varie miscele e le loro proprietà più importanti:

\begin{itemize}
	\item {$\mathbf{Acqua}$ $\mathbf{regia}$:} $3H^+Cl^-$ + $H^+NO_3^- \longrightarrow Cl_2^{\uparrow} + NO^+Cl^- + 2H_2O$ dove $NOCl$ è chiamato cloruro nitrosile. Questa miscela è in grado di attaccare anche l'oro. Funziona che il nitrico ossida il cloridrico rilasciando $Cl^-$ che attacca qualunque cosa.
	\item {$\mathbf{Dissoluzione}$ $\mathbf{in}$ $\mathbf{miscele}$ $\mathbf{di}$ $\mathbf{acidi}$ $\mathbf{con}$ $\mathbf{altri}$ $\mathbf{reagenti}$:} Vengono aggiunti altri \colorbox{yellow}{agenti che per esempio inalzano la} \colorbox{yellow}{temperatura di ebollizione dell'acido o non fanno precipitare un determinato sale.} \\\colorbox{yellow}{(es. Acido Tartarico / Acido Citrico)}.
	\begin{center}
		\begin{minipage}{0.4\textwidth}
			\includegraphics*[width=0.9\linewidth]{../images/tartarico.png}
			\captionof{figure}{Acido Tartartico}
		\end{minipage}
		\begin{minipage}{0.4\textwidth}
			\includegraphics*[width=0.9\linewidth]{../images/citrico.png}
			\captionof{figure}{Acido Citrico}
		\end{minipage}
	\end{center}
	\item {$\mathbf{Precauzioni}$ $\mathbf{nella}$ $\mathbf{digestione}$:} La digestione di composti organici genera materiale aeriforme, che ha volume maggiore rispetto all'iniziale. Esistono soluzioni come far defluire man mano che la reazione procede, o usare la digestione in bomba.
	\item {$\mathbf{Microonde}$:} Si possono usare le microonde per scaldare il campione.
\end{itemize}
 
\subsection{Fusioni:} Esistono delle bestie che rimangono insolubili in acidi minerali o che danno soluzioni instabili che tendono a precipitare. Il RESIDUO INSOLUBILE è quella porzione di campione che resta inalterato quando attaccato da acqua regia.
Il colore del residuo, dipenderà dal contenuto. (Bianco = Si, Giallo = S, Nero = Carbone, Rosso = Fe / Cr)

\paragraph{Funzionamento:} Polverizzando il campione e aggiungendo un solido acido o basico (\underline{FLUSSO} o \underline{FONDENTE}) (da 1:2 a 1:50), possiamo scaldare il tutto fino a fonderlo. Successivamente il fuso viene raffreddato e fatto a pezzi. Se tutto ha funzionato, il fuso ora di scogliera in acqua o ambiente leggermente acido.
La procedura è efficace perchè gli elettroliti portati a fusione sono molto potenti, e data l'elevata temperatura raggiunta, si può sfruttare al massimo questo effetto. 
\\Possibili \underline{SVANTAGGI} sono la possibilità di contaminazione e perdite per volatilizzazione. Può essere usato anche solo come colpo di grazia.

\paragraph{FONDENTI BASICI:} $Na_2CO_3, NaOH, KOH, Na_2O_2, CaCO_3$ Usati tendenzialmente tutti per silice o silicati 
\paragraph{FONDENTI ACIDI:} $KHSO_4, K_2S_2O_7, B_2O_3, KF + KHF_2$ Usati tendenzialmente tutti per metalli alcani da silicati, da ossidi metallici a solfati metallici o alternativi ai flussi basici	 

\subsection{Trattamenti successivi all'attacco:} Dato che il risultato di tutto questo saranno soluzioni con grandi quanitità di acido concentrato, queste potranno interferire in stadi successivi. Si possono rimuovere facendo evaporare quasi a secco e poi riprendendoli con acidi più diluiti. Attenzione a perdite, e alla difficoltà di rimozione si sali fusi.
\subsection{Digestioni di campioni organici:} 
\begin{itemize}
	\item \colorbox{yellow}{Mineralizzazione:} è il processo chimico per cui si distrugge la matrice organica per ottenere solo la concentrazione della materia inorganica.
	\item \colorbox{yellow}{Combustione:} i campioni gassosi verranno poi mandati in una colonna gas-cromatografica per la determinazione in \% delle concentrazioni (organica).
	\item \colorbox{yellow}{Incenerimento:} Si brucia la cenere fino a far volatilizzare la componente organica rimasta.
\end{itemize}	
Si tende a perdere campione in volatilizzazione.


\paragraph{Estrazioni:} Sfruttiamo l'equazione di Nernst che ci dice che il sistema chimico tenderà ad equilibrarsi qualora venisse introdotto un disturbo all'interno del sistema.
\begin{itemize}
	\item LIQUIDO - LIQUIDO: Scelta del solvente $\rightarrow$ cercheremo qualcosa con la \colorbox{yellow}{$K_d$ più grande possibile = $[A_2] / [A_1]$} dove le varie A sono le fasi liquide immiscibili dell'analita. In pratica: maggiore è la K, maggiore sarà il distacco dell'analita dalla sua matrice e lo shift verso la nuovafase composta dal solvente estrattore.
	\item LIQUIDO - SOLIDO: \colorbox{yellow}{Metodo \underline{SOXHLET}} ciclo continuo come se fosse una lava lamp in cui il sale sciolto esegue un ciclo di riscladamento e solvatazione.
	\item SOLIDO - SOLIDO: il liquido contenente il campione viene fatto passare attraverso un solido sorbente. Basato su: Van der Waals, legami H, dipolo-dipolo, ripartizione o interazione Coloumbiana. 
\end{itemize}

Ci sono poi diverse tipologie di estrazione particolari come:
\begin{itemize}
	\item \colorbox{yellow}{Estrazione accelerata:} recipiente chiuso e pressurizzato
	\item \colorbox{yellow}{Pergue and Trap:} trasporto ad una colonna cromatografica tramite gas inerte
	\item \colorbox{yellow}{Estrazione assistita da microonde:} Aiuto di orientamento utilizzando onde elettromagentiche
	\item \colorbox{yellow}{Fluidi supercritici:} si usa di solito la $CO_2$, diffusione di un gas ma capacità di trasporto di un liquido
\end{itemize}
%
Ci sarà successivamente una scelta sulla fase sorbente. Deve avere un'interazione con l'analita maggiore di quella tra analita e fase sorbente.
\\
Per finire discorso su SOLID PHASE MICROEXTRACTION (SPME): sirigna di silice su cui si accumula il campione, questo poi viene introdotto in un cromatografore che aumentando la temperatura andrà a staccare l'analita.


%----------------------------------------------------------------------------------

\newpage

\section{Analisi Volumetrica \& Misura pH}

\subsection{Principi della volumetria} 
È basata sul bilanciare il volume di una sostanza a concentrazione ignota con un'altra sostanza chiamata titolante a concentrazione nota. Quando (visivamente) si nota l'equilibrio si ricava quantitativamente la concentrazione dell'analita.
Si usa una buretta contente la soluzione ad una certa concentrazione

\subsection{Curve di titolazione}
Possiamo notare la variazione di pH quando visivamente vediamo un cambio di colore, oppure strumentalmente notiamo il flesso di una sigmoide.

\subsection{Titolazione Acido-Base}
Riporta il pH in funzione del volume di titolante aggiunto. Si avrà pH neutro nel punto equivalente solo quando si titola un acido / base forte con acido / base forte oppure entrambi deboli.
\paragraph{INDICATORI:} Acidi e basi deboli sono colorati diversamente in base al grado di protonazione $\rightarrow$ $HIn \rightleftharpoons H^+ + In^-$ e abbiamo il \underline{viraggio} quando $[HIn] = [In^-]$. Dato che l'occhio umano ha delle limitazioni noi cominciamo a vedere la differenza solo quando superiamo di 10 la concentrazione, portandoci quindi tramite un calcolo ad avere il viraggio a $pH = pK_{Hin} \pm 1$ 

\subsection{Titolazione Complessometriche} Formula generica per capirci: $$M^{n+} + L^{m-} \rightleftharpoons ML^{n-m}$$  Si può usare EDTA, e si misurerà $pM$ in funzione del volume di titolante, dove $pM = - \log_{10}{[M]}$. Sarà presente anche un tampone pH per evitare di avere precipitazioni indesiderate di analita. L'indicatore visuale di solito sono i metallocromici. Potrebbero avere range di operatività. Possono anche essere specifici e colorarsi solo quando intereagiscono con certi analiti.

\subsection{Titolazione RedOx} Formula generica per capirci: $$\text{m}Rid_1 + \text{n}Oss_2 \rightleftharpoons \text{m}Oss_1 + \text{n}Rid_2$$ Dove $Rid_1$ è l'analita e $Oss_2$ è l'ossidante. Maggiore differenza di potenziali, maggiore quantitatività. Segnalatori: Blu di metilene. Titolanti usati: 

\begin{center}
	\begin{minipage}{0.6\textwidth}
		OSSIDANTI
		\begin{itemize}
		\item Ione permanganato $MnO_4^-$ (no bisogno di indicatore)
		\item Ione dicromato $Cr_2O_7^{2-}$
		\item Cerio(IV) $Ce^{4+}$
		\item Ione bromato $BrO_3^-$
		\end{itemize}		
	\end{minipage}
	\hfill
	\begin{minipage}{0.3\textwidth}
		RIDUCENTI
		\begin{itemize}
		\item Cromo (III) $Cr^{2-}$
		\item Ferro (II) $Fe^{2-}$
		\item Ione tiosolfato $S_2O_3^{2-}$
		\item Idrazina $N_2H_4$
		\end{itemize}		
	\end{minipage}
\end{center}

\hrulefill

\subsection{Misura potenziometrica del pH:} Sono un genio, non mi serve, la so a memoria.

\newpage

\section{FORMULE}
$$M_{conc}V_{conc} = M_{dil}V_{dil}$$
\hrulefill
$$\text{Trasmittanza} = I/I_0 = e^{-Kb}$$
$$\text{Assorbanza} = -\log{(T)} = -\log{(I/I_0)} = \log{(I_0/I)} = \varepsilon_{\lambda}bC$$

Con:
\begin{itemize}
	\item $\varepsilon_{\lambda}$ : Coefficiente di estinzione molare
	\item $b$ : Cammino percorso
	\item $C$ : Concentrazione molare
\end{itemize}
\hrulefill

\paragraph{Equazione di Van Deemter: } $H=A+\frac{B}{v_m}+Cv_m$ dove:
\begin{itemize}
	\item A: Contributo cammini multipli
	\item B: Diffusione longitudinale
	\item C: Trasferimento di massa
\end{itemize}
%
Ognuna di queste funzioni è una H = f(v), ovvero che il minimo di questa equazione ci permette di stabilire la velocità ottimale per ottenre una buona risoluzione.

\hrulefill

\paragraph{Costante di equilibrio: } Data la reazione: $$aA_{g} + bB_{g} \rightleftharpoons cC_{g} + dD_{g}$$ Ottengo la costante di equilibrio:
$$K = \frac{C^cD^d}{A^aB^b}$$ Actually bisognerebbe ancora considerare che per esempio $A^a$ andrebbe ancora moltiplicato per un fattore $\gamma_A^a$ che si riferisce al coefficiente di attività

\paragraph{Capacità tampone: } $$\beta = \Delta c_{base} / \Delta_{pH} = - \Delta c_{acido} / \Delta_{pH}$$
\paragraph{Massima capacità tampone: } $$pH = pK_a = [H_3O^+] = \pm 1$$

Equazione di Nernst: $$E (\text{pot. elettrico}) = E^\circ (\text{pot. standard riduzione}) - \frac{RT}{nF} \ln\left(\frac{A^b_B}{A^a_A}\right)$$, dove A/A sarà il Quoziente di reazione

\paragraph{Stabilità dei composti: } Più ne aggiungi e meno sarà favorita. $ML_{N-1} + L \rightleftharpoons ML_N \therefore K_N=\frac{[ML_N]}{[ML_{N-1}] [L]}$. La relazione che ci saà tra $\beta$ e $K$
sarà $\beta_K = \prod_i^KK_i$. $\beta$ rappresenta la costante di formazione cumulativa, mentre $K$ quella parziale

\paragraph{Titolazione: } $C_OV_O = CV$ Questo solo se reagiscono in rapporto 1:1

\newpage
\section{Esercizi}
\paragraph{Acido Solforico: } $H_2SO_4 + H_2O \rightarrow HSO_4^- + H_3O^+ \rightleftharpoons SO_4^{--} + H_3O^+$ 
\\\\
Fun fact:
\begin{itemize}
	\item Dove $SO_4^{--}$ si chiama Solfato.
	\item Acido diprotico forte, e un acido monoprotico debole dopo la prima dissociazione
	\item Freccia singola perchè acido forte (Da ricordare a memoria)
	\item Discorso su forza acida (effetto livellante solvente e la forza di acidi)
\end{itemize}
%
Utilizzi:
\begin{itemize}
	\item Digestione umida: usato come acido ossidante (metalli che hanno un potere riducente minore dell'$H_2$)
	\item Può essere usato come ossidante + elettroliti inerti (Per aumentare la T di ebollizione dell'acido solforico)
	\item Può essere usato come essiccante (reazione con $H_2o$ estremamente esotermica)
\end{itemize}
\hrulefill
\paragraph{Acido Nitrico: } $HNO_3 + H_2O \rightarrow NO_3^- + H_3O^+$ 
\\\\
Fun fact:
\begin{itemize}
	\item Dove $NO_3^-$ si chiama Nitrato.
	\item Acido monoprotico forte
	\item Freccia singola perchè acido forte (Da ricordare a memoria)
	\item Discorso su forza acida (effetto livellante solvente e la forza di acidi)
\end{itemize}
%
Utilizzi:
\begin{itemize}
	\item Digestione umida: usato come acido ossidante (metalli che hanno un potere riducente minore dell'$H_2$)
	\item Può essere usato come ossidante + complessante ($HF$)
	\item Può essere usato come ossidante + ossidante ancora più forte per distruggere i rimasugli (Acido perclorico: $HClO_4$)
	\item Generazione acqua regia: $3H^+Cl^-$ + $H^+NO_3^- \longrightarrow Cl_2^{\uparrow} + NO^+Cl^- + 2H_2O$ dove $NOCl$ è chiamato cloruro nitrosile (complessante). Questa miscela è in grado di attaccare anche l'oro. Funziona che il nitrico ossida il cloridrico rilasciando $Cl^-$ 
	\item Con l'$Au$: $Cl^- + 3Cl\cdot + Au \longrightarrow AuCl_4^-$ ($3Cl\cdot$ è un radicale libero (neutro))
\end{itemize}
\hrulefill
\paragraph{Acido Cloridrico: } $HCl + H_2O \rightarrow Cl^- + H_3O^+$ 
\\\\
Fun fact:
\begin{itemize}
	\item Dove $Cl^-$ si chiama Cloruro.
	\item Acido monoprotico forte
	\item Freccia singola perchè acido forte (Da ricordare a memoria)
	\item Discorso su forza acida (effetto livellante solvente e la forza di acidi)
\end{itemize}
%
Utilizzi:
\begin{itemize}
	\item Digestione umida: usato come acido non ossidante ($H^+$ è la specie più ossidante)
	\item Può ossidare le specie chimiche con potere riducente maggiore di $H_2$ come per esempio $Zn$
	\item Generazione acqua regia: (Vedi sopra)
	\item Composto non idrosolubile genere un cloruro che invece è idrosolubile
	\item Acido + Agente Ossidante ($KClO_3$)
\end{itemize}
\hrulefill
\paragraph{Acido Fluoridrico: } $HF + H_2O \rightleftharpoons F^- + H_3O^+$ 
\\\\
Fun fact:
\begin{itemize}
	\item Dove $F^-$ si chiama Fluoruro.
	\item Acido monoprotico forte
	\item Doppia freccia perchè non si dissocia comunque completamente in acqua perchè piccolo e stabile (Da ricordare a memoria)
	\item Discorso su forza acida (effetto livellante solvente e la forza di acidi)
\end{itemize}
%
Utilizzi:
\begin{itemize}
	\item Acido non ossidante
	\item Complessante
	\item Può dare interferenze nelle misure successive
\end{itemize}
\hrulefill
\paragraph{Acido Fosforico: } $H_3PO_4 + H_2O \rightleftharpoons H_2PO_4^- + H_3O^+ \rightleftharpoons HPO_4^{--} + H_3O^+ \rightleftharpoons PO_4^{3-} + H_3O^+$
\\\\
Fun fact:
\begin{itemize}
	\item Dove $PO_4^{3-}$ si chiama Fosfato.
	\item Acido triprotico debole
	\item Discorso su forza acida (effetto livellante solvente e la forza di acidi)
\end{itemize}
%
Utilizzi:
\begin{itemize}
	\item Digestione umida: usato come acido non ossidante (probabilmente perchè acido debole)
	\item Poco usato perchè molti fosfati metallici sono poco idrosolubili
	\item Può dare interferenze nelle misure successive
\end{itemize}
\hrulefill
\paragraph{Acido Perclorico: } $HClO_4 \rightarrow ClO_4^- + H_3O^+$
\\\\
Fun fact:
\begin{itemize}
	\item Dove $CLO_4^-$ si chiama perclorato
	\item Quando reagisce con un metallo forma il perclorato di quel metallo (molti sono solubili)
\end{itemize}
%
Utilizzi:
\begin{itemize}
	\item Usato con agenti ossidanti
	\item Usato con complessanti
	\item È esplosivo, quindi usato alla fine quando poco reagente
\end{itemize}


\newpage
\section{Domande}
\begin{center}
	\includegraphics*[width=0.9\linewidth]{../images/wtf1.png}
	\captionof{figure}{Perchè $[Ag^+] = 2s?$ Se devo considerare la doppia presenza di moli, non ne tengo già in considerazione in $(2s)^2$?}
\end{center}

\hrulefill

\paragraph{Cos'è un metallo che si passiva?} Metallo che si ossida solo superficialmente

\hrulefill

\newpage

\section{Tabelle}

\subsection{Solventi}
\begin{center}
\begin{tabular}{l|l|l}
	\toprule
	Tipologia Solventi & Opzione 1 & Opzione 2 \\
	\midrule
	Polari & \colorbox{yellow}{$H_2O$} & - \\
	Apolari & \colorbox{yellow}{$CH_3COOH$} & - \\
	Protofili & \colorbox{yellow}{$NH_3$} & - \\
	Protogeni & \colorbox{yellow}{$CH_3COOH$} & $H_2SO_4$ \\
	Aprotici & $CH_3CN$ & Benzene / \colorbox{yellow}{Toluene} \\
	Anfiprotici & \colorbox{yellow}{$H_2O$} & $CH_3OH$ \\
	\bottomrule
\end{tabular}
\end{center}

\subsection{Leganti}
\begin{center}
\begin{tabular}{l|l}
	\toprule
	Monodentati & Polidentati \\
	\midrule
	\colorbox{yellow}{$Br^-$} & \colorbox{yellow}{Etilendiammina} \\
	\colorbox{yellow}{$Cl^-$} & IDA \\
	$CN^-$ / $SCN^-$ & NTA \\
	\colorbox{yellow}{$H_2O$} & \colorbox{yellow}{EDTA} \\
	\colorbox{yellow}{$NH_3$} & DCTA \\
	\bottomrule
\end{tabular}
\end{center}

\begin{center}
\begin{tabular}{l|l|l}
	\toprule
	Durezza & Metalli accettori & Leganti \\
	\midrule
	HARD (piccola nuvola elettronica) & \colorbox{yellow}{$Au^{3+}$}, $Fe^{3+}$ & \colorbox{yellow}{$CH_3COO^-$ (ione acetato)}, \colorbox{yellow}{$NH_3$}, \colorbox{yellow}{$Cl^-$} \\
	BORDER LINE & $Fe^{2+}$, $Cu^{2+}$ & $NO_2^-$, Imidazolo \\
	SOFT (grande nuvola elettronica)& $Cu^+$, \colorbox{yellow}{$Au^+$} & \colorbox{yellow}{$CN^-$ (ione cianuro)}, \colorbox{yellow}{$SCN^-$ (ione tiocianato)}\\
	\bottomrule
\end{tabular}
\end{center}

\subsection{Idrosolubilità}
\colorbox{yellow}{SOS Capellino Francesca, Alan ha perso.}
\begin{center}
\begin{tabular}{l|l}
	\toprule
	Molto idrosolubili & Poco idrosolubili \\
	\midrule
	\colorbox{yellow}{Alogenuri:} Cloruri, Bromuri, Ioduri. NO $(Ag, Hg, Pb, Tl)^{1\vee2+}$ &\colorbox{yellow}{ Solfuri} \\
	\colorbox{yellow}{Ammoniati} & \colorbox{yellow}{Carbonati} \\
	\colorbox{yellow}{Acetati} e \colorbox{yellow}{perclorati} (Tranne IA) \colorbox{yellow}{Fosfati} \\
	\colorbox{yellow}{Nitrati} & \colorbox{yellow}{Ossalati} \\
	\colorbox{yellow}{Solfati} del gruppo IA e $(Mg, Fe)^{2+}$ & \colorbox{yellow}{Solfati $(Ca, Sr, Ba, Pb, Hg, Ag)^{2+}$} \\
	\bottomrule
\end{tabular}
\end{center}

\subsection{Pretrattamenti}
\paragraph{Digestione umida:}
\begin{center}
\begin{tabular}{l|l}
	\toprule
	Acidi ossidanti & Acidi non ossidanti \\
	\midrule
	\colorbox{yellow}{$HNO_3$} & \colorbox{yellow}{$HCl$} \\
	\colorbox{yellow}{$H_2SO_4$} & \colorbox{yellow}{$HF$} \\
	\colorbox{yellow}{$HClO_4$} & \colorbox{yellow}{$H_3PO_4$} \\
	\bottomrule
\end{tabular}
\end{center}

\paragraph{Digestione umida soluzioni:}
\begin{center}
\begin{tabular}{l|l}
	\midrule
	\colorbox{yellow}{Complessante + acido ossidanti} & $HF + HNO_3$, $H_2SO_4$, $HClO_4$ \\
	\colorbox{yellow}{Acqua regia} & $HNO_3 + 3HCl$ \\
	Acidi + agenti ossidanti & \colorbox{yellow}{$HCl + (H_2O_2, Br_2, KClO_3)$}  \\
	Acidi + elettroliti inerti &  $H_2SO_4 + (Na_2SO_4$, $K_2SO_4$, $(NH_4)_2SO_4)$\\
	Acido organico e \colorbox{yellow}{complessanti} &  \colorbox{yellow}{Acido tartartico, citrico, ossalico, EDTA}\\
	Acidi + catalizzatori &  $(Cu, Hg)^{2+}$\\
	\bottomrule
\end{tabular}
\end{center}

\paragraph{Fusione:}
\begin{center}
\begin{tabular}{l|l}
	\toprule
	Fondenti basici & Fondenti acidi \\
	\midrule
	$Na_2CO_3$, \colorbox{yellow}{$NaOH$}, \colorbox{yellow}{$KOH$}, $CaCO_3 + NH_4Cl$ & $K_2S_2O_7$ \\
	$Na_2B_4O_7$, $Na_2CO_3$ & \underline{$B_2O_3$} \\
	$Na_2O_2$ (attacco solfuri) & \colorbox{yellow}{$KF + KHF_2$} \\
	\bottomrule
\end{tabular}
\end{center}

\subsection{ESTRAZIONI}
\paragraph{Solventi liq-liq: } \colorbox{yellow}{$H_2O$, etere dietilico, $CH_2Cl_2$ (diclorometano)}
\paragraph{Solventi sol-liq: }
\begin{center}
\begin{tabular}{l|l|l}
	\toprule
	NON POLARE & POLARE & SCAMBIO IONICO \\
	\midrule
	\colorbox{yellow}{Silice - C8} & Si-CN & SCX (scambio cationico con $SO_3^-$) \\
	\colorbox{yellow}{Silice - C18} & \colorbox{yellow}{Silice} & (scambio anionico con \underline{chelanti}) \\
	Silice - fenile & \colorbox{yellow}{Allumina} & SAX (scambio anionico con gruppi ammonio) \\
	\bottomrule
\end{tabular}
\end{center}

\subsection{TITOLAZIONI}
\begin{center}
\begin{tabular}{l|l|l}
	\toprule
	Titolazione & Titolante & Indicatore \\
	\midrule
	Acido - Base & \colorbox{yellow}{$NaOH$}, $KOH$, \colorbox{yellow}{$HCl$}, $HNO_3$ & \colorbox{yellow}{Fenolftaleina}, rosso di cresolo, metilarancio \\
	Complessiometrica & $CN^-$, $Ag^+,$ $Hg^{2+}$, \colorbox{yellow}{EDTA} & \colorbox{yellow}{Nero eriocromo}, $SCN^-$ \\
	Di precipitazione & $Ag^+$ & \\
	RedOx (ox) & \colorbox{yellow}{$MnO_4^-$ (permanganato)}, $I_2$, $Ce^{4+}$, \colorbox{yellow}{$Cr_2O_7^{2-}$ (dicromato)} & \colorbox{yellow}{Blu di metilene}, $SCN^-$ \\
	RedOx (red) & \colorbox{yellow}{$Fe^{2+}$}, \colorbox{yellow}{$Cr^{2+}$}, \colorbox{yellow}{$N_2H_4$ (idrazina)}, \colorbox{yellow}{$S_2O_3^{2-}$ (tiosolfato)} & \colorbox{yellow}{Blu di metilene}, $SCN^-$ \\
	\bottomrule
\end{tabular}
\end{center}

\newpage

\section{NOMENCLATURA COMPOSTI}

\paragraph{Acidi}
\begin{itemize}
	\item $H_2SiF_6$: Acido Esafluorosilicico
	\item $H_3BO_3$: Acido Borico
	\item $HBF_4$: Acido tetrafluoroborico
	\item $H_2SnO_3$: Acido metastannico
\end{itemize}

\paragraph{Sali}
\begin{itemize}
	\item $NOCl$: Cloruro di nitrosile
\end{itemize}

\paragraph{Ioni (Anioni)}
\begin{itemize}
	\item $Fe_2O_4^{2-}$: Ferrito
	\item $CrO_3^{3-}$: Cromito
\end{itemize}

\paragraph{Solventi organici}
\begin{itemize}
	\item $CH_2Cl_2$: Dicoloro metano
	\item Metanolo
	\item Acido acetico
	\item Benzene 
	\item Toluoene
\end{itemize}

\paragraph{Ossidi}
\begin{itemize}
	\item $Cr_2O_3$: Ossido cromoso
	\item $Na_2O_2$: Perossido di sodio
	\item $Na_2B_4O_7 + 10H_2O$: Tetraborato di sodio decaidrato / Borace
	\item $KHSO_4$ Bisolfato di potassio
	\item $K_2S_2O_7$ Pirosolfato di potassio
	\item $B_2O_3$ Ossido borico
\end{itemize}

\begin{center}
	\includegraphics*[width=0.9\linewidth]{../images/doodles.jpg}
	\captionof{figure}{Doodles}
\end{center}

\section{Laboratorio}

\subsection{$Cu$ in ottone}
\paragraph{Pretrattamento:} Stimare la quanitità di Cu nell'ottone e pesare il campione. Successivamente l'abbiamo attaccato con acido (Nitrico per la precisione per poter ossidare il Cu).
\begin{itemize}
	\item Poco concentrato: $Cu + 2NO_3^- + 4H^+ \rightarrow 2NO_2^\uparrow + Cu^{2+} + 2H_2O$ 
	\item Tanto concentrato: $3Cu + 2NO_3^- + 8H^+ \rightarrow 2NO^\uparrow + 3Cu^{2+} + 4H_2O$
	\item Reazione dello zinco: $4Zn + NO_3^- + 10H^+ \rightarrow 4Zn^{2+} + NH_4^+ + 3H_2O$
\end{itemize}
%
Cominciamo con la preparazione degli standard. Abbiamo una stima della molarità, quindi possiamo stimare l'assorbanza del campione, conoscendo il $\varepsilon$ del $Cu$ e il cammino ottico della provetta. Sapendo il valore teorico possiamo preparare 4 standard esterni utilizzando varie concentrazioni note di $Cu$, in modo tale che il valore incognito cada in quel range di assorbanze.
\\\\
Per farlo usiamo le formule inverse per ricavare la concentrazione degli standard preparate da uno standard primario. Partiamo sempre da: $\varepsilon b c = A$
\\\\
Abbiamo arrotondato i volumi e quindi ricalcolato le concentrazioni per semplificarci la vita, poi abbiamo aggiunto una goccia di $HNO_3$ per simulare meglio la matrice, e infine abbiamo portato tutto a volume.

\paragraph{Misura con spettrofotometro UV-Vis:} Dopo aver analizzato tutti e 5 (4 standard e il campione) i campioni abbiamo una $\lambda$ alla quale corrispondeva il picco più intenso e l'abbiamo usato come riferimento per creare la nostra retta di taratura. Il grafico della retta è assorbanza in funzione della concentrazione.

\paragraph{Analisi dati:} Il valore ottenuto è leggermente più basso di quello che ci aspettavamo. Infatti come ultimo passaggio dovremmo ricalcolare l'assorbanza usando il nuovo coefficiente di estinzione molare calibrato (il coefficiente angolare). Con questa modifica si ottiene un valore molto più prossimo.

\subsection{$Ni$ in acciaio}
\paragraph{Pretrattamento:} Peso campione. Pulito i matracci con $HNO_3$ per evitare contaminazioni da parte di residui presenti nella vetreria. Dopodichè abbiamo messo tutto nella bomba teflon per microonde ad eseguire la digestione. Infine abbiamo filtrato i residui con acqua ultrapura.

\paragraph{Preparazione standard:} Partiamo con la preparazione del bianco dei reagenti, questo sarà composto da acqua regia. Questo per vedere se i reagenti davano qualche interferenza nella misura (anche questo è finito in microonde perchè tutto deve subire lo stesso processo del campione). Per preparare lo standard interno abbiamo usato una soluzione ad 1ppm di $Lu$, e l'abbiamo aggiunta al campione, agli standard esterni e al bianco degli standard.
\\\\
La situazione finale dovrebbe essere:
\begin{itemize}
	\item Campione: Lametta, Acqua Regia, Lu $\rightarrow$ microonde
	\item Bianco reagenti: Acqua Regia $\rightarrow$ microonde
	\item Bianco standard: Acqua Regia, Lu
	\item Standard esterni (4): Acqua regia, conc. var. di Ni, Lu 
\end{itemize}

\paragraph{ICP:} In ordine di analisi:
\begin{itemize}
	\item Bianco degli standard
	\item Standard esterni (4) (da conc. minore a maggiore)
	\item ------ PULIZIA CON $HNO_3$ ------ 
	\item Bianco dei reagenti
	\item Campione
\end{itemize}
%
Per avere tutti i dati corretti abbiamo sottratto il bianco dei reagenti dal campione (entrambi sono stati nel microonde).

\paragraph{Analisi dati:} Per usare concentrazione si usa $mol / L$, se invece si vogliono usare i ppm si usa $massa / L$. Dalla retta di taratura degli std. ext. otteniamo la concentrazione di Ni nella soluzione, dalla retta di taratura data dal rapporto delle concentrazioni di Ni / Lu degli standard esterni (combinazione dei 2 standard). Il segnale del Lu corrisponde al segnale del bianco degli standard.

\subsection{Fenolo / Cresolo}
\paragraph{Calibrazione HPLC:} Abbiamo usato uno standard fornito dalla prof per creare la retta di taratura che poi abbiamo usato per il nostro campione incognito.

\end{document}