\chapter{Conclusions}

The aim of this study was to reconstruct the tip shape via \textit{'ray tracer method'}, starting from the generation, via Python module, of the ideal topography of the tip characteriser. Then the tip shape reconstruction was made by “eroding” the real measurement with the generated structure.
More in detail, the \textit{ray tracer} approach allows to generate multiple models of the expected sizes and orientations, then to analyse them by means of image recognition algorithm and finally, to proceed with the eroding algorithm.

In particular in this work the complete implementation of a spherical support was performed, as a crucial starting point for the development of more complex structures. This means that, given the images of nanospheres, we can reconstruct AFM tips, which is the main step for the deconvolution of AFM topographies. The same approach can be easily scaled to other shapes, but at this time a full-automatic algorithm generates all the intermediate steps. Once the simulation is finished, it is possible to obtain statistics on how well the program performed compared to more established methods, thus providing insight into a further improvement.

In conclusion, we can state that the ability to simulate ideal measurements and accurately to reconstruct tip shapes using ray tracing techniques opens up new possibilities for improving the precision and speed of tip reconstruction algorithms in AFM studies. 

\vspace{10pt}

As a further perspective, studies on different tip shapes will benefit from the groundwork established by the sphere implementation. More in detail, we will focus on the $TiO_2$ nanosheets, that, due to their unique properties and challenging applications, are widely investigated in material science \cite{scarano}.
