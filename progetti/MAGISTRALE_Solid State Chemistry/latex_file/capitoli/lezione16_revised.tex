\section{27/11/24}

\subsection{[$\checkmark$] LAB day 1 (27/11/24)}

\subsubsection{Part 1: Preparation}

\paragraph{Objective 1:} Prepare 55 mL of solution, which needs to be:
\begin{itemize}
    \item 10 [wt\%] in glacial acid
    \item 0.1 [M] of \ce{[Bi(NO_3)_3 5H_2O]}
\end{itemize}  

\noindent Knowing that the density of acetic acid is $\sim 1$ [Kg/L] we can calculate that we need \textbf{5.5 [mL]} to create $55$ [mL] of solution to match the $10$ [wt\%] . To do this we just add to the $5.5$ [mL] of acetic acid the remaining water.

\vspace{10pt}

\begin{minipage}{0.48\textwidth}
    \centering
    \includegraphics[width=0.6\linewidth]{../Group_5/photo_report/IMG20241127152932.jpg}
    \captionof{figure}{5.5mL of acetic acid}
\end{minipage}%
\begin{minipage}{0.48\textwidth}
    \centering
    \includegraphics[width=0.6\linewidth]{../Group_5/photo_report/IMG20241127153255.jpg}
    \captionof{figure}{55mL of solution}
\end{minipage}

\vspace{10pt}

\noindent For the final part, we know that the MW of \ce{[Bi(NO_3)_3 5H_2O]} is $485.07$ [g/mol]. 

\noindent This means that to obtain $55$ [mL] of solution at $0.1$ [M], we will need to add \textbf{2.67 [g]} of powder.

\begin{figure}[ht]
    \centering
	\includegraphics*[width=0.23\linewidth]{../Group_5/photo_report/IMG20241127151359.jpg}
    \caption{2.67 [g] of Bismuth powder}
\end{figure}

\noindent The final solution needs to be $55$ [mL] at $0.1$ [M] of Bismuth-complex, which equals to $0.1$ [M] * $55$ [mL] $= 0.0055$ [mol] of powder added. 

\noindent We can then convert to grams with $0.0055$ [mol] * $485.07$ [g/mol] = $2.67$ [g].


\begin{figure}[ht]
    \centering
	\includegraphics*[width=0.25\linewidth]{../Group_5/photo_report/IMG20241127153705.jpg}
    \caption{magnetic stirring of the 55 [mL] solution of Bi-complex + acetic acid}
\end{figure}

\vspace{10pt}

\paragraph{Objective 2:} Prepare 33 mL of solution, which needs to be:
\begin{itemize}
    \item $0.1$ [M] of \ce{KBr}
\end{itemize}

\noindent To prepare the \ce{KBr} solution we had to find how much grams we need to add to water.

\noindent We multiplied the MW of \ce{KBr} with the final volume and the molarity required:
$$33*10^{-3} \text{[L]} * 0.1 \text{[mol/L]} * 119.002 \text{[g/mol]} \approx \textbf{0.3927 [g]} $$

\begin{figure}[ht]
    \centering
	\includegraphics*[width=0.25\linewidth]{../Group_5/photo_report/IMG20241127152740.jpg}
    \caption{Powder of KBr}
\end{figure}

\paragraph{Execution steps:}
\begin{itemize}
    \item Weight the powder
    \item Create the solution of water and acetic acid
    \item Add the powder to the solution
    \item Magnetic stirring until a complete dissolving
    \item Add \ce{KBr} (Droplet method) 
    \item Keep stirring for 24h
\end{itemize}

\begin{minipage}{0.32\textwidth}
    \centering
	\includegraphics*[width=0.9\linewidth]{../Group_5/photo_report/IMG20241127153705.jpg}
    \captionof{figure}{magnetic stirring}
\end{minipage}%
\begin{minipage}{0.32\textwidth}
    \centering
	\includegraphics*[width=0.9\linewidth]{../Group_5/photo_report/IMG20241127154319.jpg}
    \captionof{figure}{setup for drops}
\end{minipage}%
\begin{minipage}{0.32\textwidth}
    \centering
	\includegraphics*[width=0.9\linewidth]{../Group_5/photo_report/IMG20241127155233.jpg}
    \captionof{figure}{added first drops}
\end{minipage}

\begin{figure}[ht]
    \centering
	\includegraphics*[width=0.25\linewidth]{../Group_5/photo_report/IMG20241128135636.jpg}
    \caption{Solution without P25 after 24h}
\end{figure}

\subsubsection{Part 2: Preparation}

\paragraph{Objective 1 and 2:}

Same procedure, but with different quantites:
\begin{itemize}
    \item \textbf{30 [ml] of total solution (10 [wt\%] acetic acid) at 0.03 [M] of Bi-comples}
    \item \textbf{20 [mL] of Kbr at 0.025 [M]}
\end{itemize}  

\noindent Which means that the reagents needed are:

\begin{itemize}
    \item \textbf{0.44 [g] of Bi}
    \item \textbf{3 [mL] of acetic acid}
    \item \textbf{0.0597 [g] of KBr}
    \item \textbf{1 [g] of P25 (Titania)}
\end{itemize}

\begin{minipage}{0.32\textwidth}
    \centering
    \includegraphics*[width=0.9\linewidth]{../Group_5/photo_report/IMG20241127155631.jpg}
    \captionof{figure}{Powder of Bi-complex}
\end{minipage}%
\begin{minipage}{0.32\textwidth}
    \centering
    \includegraphics*[width=0.9\linewidth]{../Group_5/photo_report/IMG20241127162709.jpg}
    \captionof{figure}{KBr weighted}
\end{minipage}%
\begin{minipage}{0.32\textwidth}
    \centering
    \includegraphics*[width=0.9\linewidth]{../Group_5/photo_report/IMG20241127161103.jpg}
    \captionof{figure}{P25 weighted}
\end{minipage}

\begin{figure}[ht]
    \centering
	\includegraphics*[width=0.45\linewidth]{../Group_5/photo_report/IMG20241127155910.jpg}
    \caption{Final solution of acetic acid and Bi-complex (30 [mL])}
\end{figure}

\newpage

\paragraph{Execution steps:}

\begin{itemize}
    \item Weight the powder
    \item Create the solution of water and acetic acid
    \item Add the powder to the solution
    \item Magnetic stirring until a complete dissolving
    \item Add the 1 [g] of P25 (Titania)
    \item Add \ce{KBr} (Droplet method) 
    \item Keep stirring for 24h
\end{itemize}

\begin{figure}[ht]
    \centering
	\includegraphics*[width=0.25\linewidth]{../Group_5/photo_report/IMG20241128135643.jpg}
    \caption{Solution with P25 after 24h}
\end{figure}

\begin{minipage}{0.48\textwidth}
    \centering
    \includegraphics[width=0.6\linewidth]{../Group_5/photo_report/IMG20241127161909.jpg}
    \captionof{figure}{Main solution + P25}
\end{minipage}%
\begin{minipage}{0.48\textwidth}
    \centering
    \includegraphics[width=0.6\linewidth]{../Group_5/photo_report/IMG20241127163549.jpg}
    \captionof{figure}{Adding KBr to the solution + P25}
\end{minipage}

\newpage

\subsubsection{LAB-Theory: Reaction of the $Bi$} 

The reason why we work in an acid ambient is because the \ce{Bi^3+} ion tends to precipitate as \ce{Bi2O3}. 

\noindent The acid environment prevents this, leaving the \ce{Bi^3+} as an ion in the solution.

\noindent Whenever the \ce{KBr} is added, it dissolves, binding the \ce{Br-} to the \ce{Bi^3+}. 

\noindent Due to the strong force of the \ce{Bi^3+}, an \ce{O^2-} will bind itself to the molecule. 

\noindent In this LAB we don't have the capability to avoid it. 

\noindent Once it precipitates, we can filter it, removing the water, acetic acid and the potassium nitrate. 

\noindent In the second experiment with the P25, the precipitate binds to the P25, which remained inert during the reaction. 

\noindent This is a great way to precipitate the \ce{BiOBr} on a substrate.

\subsubsection{LAB-Theory: Why droplet method?} 

The reason why we use the droplet method to add the \ce{KBr} is to dissolve as much as possible the salt.
Since the salt tends to form conglomerates in the solution, adding low quantites of it helps improving the dissolution.
This gives us the possibility to increase the yield of the reaction without wasting additional \ce{KBr}.

\newpage

\subsubsection{Constants:}

\begin{itemize}
    \item Bismuth nitrate pentahydrate MW = 485.07 [g/mol]
    \item Potassium bromide MW = 119.002 [g/mol]
\end{itemize}

% 15:13 Powder of Bismuto
% 15:27 Powder of KBr
% 15:29 5.5mL of acetic acid
% 15:32 55mL of solution
% 15:37 magnetic stirring of Bi + acetic acid
% 15:43 setup for drops
% 15:52 added first drops
% 15:56 Powder of Bismuto
% 15:59 New solution of acetic acid (Part 2)
% 16:11 P25 pesato
% 16:19 Acetic acid + P25
% 16:27 KBr weighted
% 16:35 Adding KBr to acetic acid + P25
