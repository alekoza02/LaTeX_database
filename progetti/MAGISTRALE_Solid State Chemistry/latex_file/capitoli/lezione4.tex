\section{9/10/24}

\subsection{[$\checkmark$] Colloids and Sol-Gel}

Formation of an oxide network through polycondensation.

\paragraph{Definition of colloids: } Mix of 2 materials (a scattering and a scattered) where one of them is so small (nanoparticles) that not even gravity affects them. It will not generate a precipitation. Examples can be: Foam, Aerosol, Emulsion, Smoke, Sol. A colloid can be detected using the Tyndall effect.

\paragraph{Sol-Gel method parameters}
\begin{itemize}
    \item Kind of solvent (polar solvent helps getting the product faster)
    \item Temperature
    \item pH (controls the speed of the reaction)
    \item Concentration of solvent
    \item Steric factor of the ligands
\end{itemize}

\paragraph{GEL} The most diffuse type of GEL is the chemical Gel, obtained using the polimerization of metallorganic monomer in alchoolic solution or in water solution, this is an irreversible reaction. Through hydrolysis we can split the metal from the organic part (which forms an $-OH$ and an $-M(OR)$) and then using condensation we can bound them together (see later).

\paragraph{PRO - CONS}

PRO:
\begin{itemize}
    \item Cheap
    \item Low temperature
    \item High purity (of the product material)
    \item molecular structure control
    \item High homogeneity (similar size)
    \item Different shapes
    \item Nanostructures
\end{itemize}

CONS:
\begin{itemize}
    \item Expensive reagents
    \item Shrinkage during essication
    \item Residual pores and impurities (from the matrix)
    \item Long time process
\end{itemize}

\paragraph{hydrolysis and condensation:} Mi fido, the key difference is that in acidic environments we can shift the reaction's balance towards the product, facilitating the hydrolysis of the $SiOR$. It's a true catalyst, since we obtained it as a product ($H^+$). While using a basic environment the condensation is facilitated. The catalyst is once again obtained as a product.

\paragraph{Difference between Acid / Basic catalyzed reaction: }
The morphology changes, depending on the final need:
\begin{itemize}
    \item Acid: more straight chains
    \item Basic: more branched networks
\end{itemize}

\paragraph{$TiO_2$ preparation: } It is prepared in acidic solution. We combine $Ti(OC_3H_7)_4$ in isopropylic alchool. To remove the solvent, it is sufficient to dry at $70^{\circ}C$ one night and 2 hours at $500^{\circ}C$ to eliminate the remaining organic part.

\subsection{[$\checkmark$] Film Formation}

You can do it by dip coating (la sai), spin coating (la sai), sputtering (la sai) and chemical vapor deposition (la sai).
