\section{02/12/24}

\subsection{[$\checkmark$] LABORATORIO giorno 3 (2/12/24)}

\subsubsection{Theory UV-Vis: }

The instrument has a range of (800-200 [nm]), this covers completely the visible spectrum and a portion of UV and NIR. 

The reason we used the tool in reflectance mode is beacause our sample was not transperent since it was a powder.

We could have created a liquid solution to use the absorbance mode, but it would have altered the crystalline structure.

The objective of this measurement was to obtain the value of the energy gap.

The issue is that reflectance cannot be directly used instead of the absorbtion. Using the Kubelka-Munk function we can transform the reflectance into a function that is directly proportional to the absorbance.

Here's the formula: 

$$\frac{K(\lambda)}{S(\lambda)} = \frac{(1-R_{\infty})^2}{2R_{\infty}}$$

From here we will have to transform the absorbance into a Tauc plot, which will enhance the precision in obtaining the value of the energy gap. Essentially we raise the whole function to a power, which depends on the the type of the energy gap (direct - indirect) and if it's prohibited or not.

We think that it is a prohibited band gap transition because from the DOS map we see a transition from the 2p orbitals of the O and Br to the 6p of Bi. This means that the there's no change in the l quantum number and therefore it's prohibited. But, at the end, it's still allowed by other factors that "soften" the selection rule. 


So, the BiOBr is a \textbf{INDIRECT} and \textbf{ALLOWED} energy gap. Which means that to obtain the final value we'll have to "correct" the Kubelka-Munk with this formula:

$$(F(R)h\nu)^{\frac{1}{2}}\propto h\nu - E_{gap}$$
