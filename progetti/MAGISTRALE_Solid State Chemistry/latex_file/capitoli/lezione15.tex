\section{27/11/24}

\subsection{[$\checkmark$] Maurino - Health science}

When there's an excited molecole, it cannot transfer energy to other molecules by collision, due to to a NON overlapping orbitals. But this can still happen in other ways, generating a singlet state in the Oxygen.

$$S^*(T) + O_2(T) \rightarrow S + O_2^*(S)$$

Considering the Couloumb repulsion we have a repulsion and a well of probability where the electron cannot overlap in the same position, BUT there's an exchange interaction (different spins), gives the possibility for two electrons to occupy the same space. This happens only if the spin is different. This means that on average two electrons with same spin are more distant than two with different spin, this means that the triplet state where electrons have different spins are more energetic. But still, singlets of higher level have a higher energy. This is because the reaction can happen only if two singlet state combine with eachother, otherwise it would require to much energy.

\paragraph{Quantum efficiency:} Given $R + h\nu \rightarrow R^* \rightarrow P | R + h\nu$, the quantum efficiency is the fraction of number of P over the absorbed photons (how many time the reaction started).

\paragraph{$CO_2$ problem:} $CO_2$ sequestration can be done. Afterwards we can storage it inside impermable rock formation which where previously used for methane extraction.

\paragraph{Energy of molecules:} The different energy needed to promote electrons can dissolve the molecule. This happens because giving more energy to electrons, they change orbitals, effectevily dissolving previously generated bonds (for example $\sigma$ and $\pi$). This happens because the second Condon-Morse curve generated may have a small concavity.