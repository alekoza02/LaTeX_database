\section{09/12/24}

\subsection{[$\checkmark$] LABORATORIO giorno 6 (9/12/24)}

\subsubsection{ATR \& Raman: }

\newpage
\paragraph{First spectrum ATR of $Bi(NO_3)_3$:}

This is the starting reagent, we can see that there are a lot of signals, because we are analyzing solids. Considerations:

\begin{itemize}
    \item The wide peak at ~3000 $cm^{-1}$ is due to the stretching of the $OH$ bond in the water.
    \item At around ~1500 and ~1250 $cm^{-1}$ we have the stretching of the $NO$ bond.
\end{itemize}

\begin{figure}[ht]
    \centering
	\includegraphics*[width=0.7\linewidth]{../Group_5/grafici/Bi(NO3)3_atr_base.png}
    \caption{$Bi(NO_3)_3$ ATR}
\end{figure}

\newpage
\paragraph{First spectrum ATR of Acetic Acid:}

\begin{itemize}
    \item The CH peak is hidden behind the OH stretching at ~3000 $cm^{-1}$.
    \item The high peak at ~1700 $cm^{-1}$ is due to the stretching of the $C=O$.
    \item The peak at ~1400 $cm^{-1}$ is due to the complex of $CO$ + $OH$.
    \item The peak at ~1300 $cm^{-1}$ is due to the stretching of the $CO$.
\end{itemize}

\begin{figure}[ht]
    \centering
    \includegraphics*[width=0.7\linewidth]{../Group_5/grafici/Acetic_acid_base.png}
    \caption{Acetic acid ATR}
\end{figure}

\newpage
\paragraph{First spectrum ATR of P25:}

\begin{itemize}
    \item The wide peak below the ~1000 $cm^{-1}$ is attribueted to $Ti-O$.
    \item The peak at ~3400 $cm^{-1}$ is attribueted to $OH$ of water.
    \item The peak at ~1700 $cm^{-1}$ is attribueted to $C=O$ of $CO_2$ ??? Fra richiesta 24/12.
\end{itemize}

\begin{figure}[ht]
    \centering
    \includegraphics*[width=0.7\linewidth]{../Group_5/grafici/P25_basic.png}
    \caption{P25 ATR}
\end{figure}

\newpage
\paragraph{First spectrum ATR of BiOBr:}

\begin{itemize}
    \item The first peak at 3500 $cm^{-1}$ is probably due to non coordinated oxygen on the surface
    \item the peak around 1400 $cm^{-1}$ is propably due to nitrates, but we will confirm that with Raman
    \item the last few peaks at 500 $cm^{-1}$ are due to the BiOBr presence.
\end{itemize}

\begin{figure}[ht]
    \centering
    \includegraphics*[width=0.7\linewidth]{../Group_5/grafici/BiOBR.png}
    \caption{BiOBr ATR}
\end{figure}

\newpage
\paragraph{First spectrum ATR of BiOBr @ TiO2:}

\begin{itemize}
    \item The small peak at ~1300-1500 $cm^{-1}$ is attribueted to residues of nitrates.
    \item The peaks lesser than 1000 $cm^{-1}$ are bulk TiO2.
    \item At 3500 $cm^{-1}$ there's OH stretching.
    \item At 2300 $cm^{-1}$ there's CO2.
\end{itemize}

\begin{figure}[ht]
    \centering
    \includegraphics*[width=0.7\linewidth]{../Group_5/grafici/BiOBR_TIO2.png}
    \caption{BiOBr ATR}
\end{figure}

\newpage
\paragraph{First spectrum Raman of $Bi(NO_3)_3$:}

\begin{itemize}
    \item Around ~1000 $cm^{-1}$ we see a signal by the nitrate group.
    \item The other high peak at ~100 $cm^{-1}$ is related to Bi-O bond. 
\end{itemize}

\begin{figure}[ht]
    \centering
    \includegraphics*[width=0.7\linewidth]{../Group_5/grafici/BiNO33_raman.png}
    \caption{$Bi(NO_3)_3$ Raman}
\end{figure}

\newpage
\paragraph{First spectrum Raman of $BiOBr$:}

\begin{itemize}
    \item here the peak related to the nitrate is no more visible, hence unknown peaks in the ATR spectra cannot be nitrates related, but probably there are organic contaminations; we can confirm this by observing FESEM images, where we found an organic agglomerate not visible in XRD
    \item probably there are organic contaminations, the peaks in the IR are caused by them. 
\end{itemize}

\begin{figure}[ht]
    \centering
    \includegraphics*[width=0.7\linewidth]{../Group_5/grafici/BiOBr_Raman.png}
    \caption{$BiOBr$ Raman}
\end{figure}


\newpage
\paragraph{First spectrum Raman of $P25$:}

\begin{itemize}
    \item Only peaks related to anatase are visible because anatase has a bigger concentration
    \item Anatase is extremely active at IR and Raman.
\end{itemize}

\begin{figure}[ht]
    \centering
    \includegraphics*[width=0.7\linewidth]{../Group_5/grafici/P25_raman.png}
    \caption{$P25$ Raman}
\end{figure}


\newpage
\paragraph{First spectrum Raman of $P25+BiOBr$:}

\begin{itemize}
    \item peaks of p25 are prevalent
\end{itemize}

\begin{figure}[ht]
    \centering
    \includegraphics*[width=0.7\linewidth]{../Group_5/grafici/BiOBr_TiO2_raman.png}
    \caption{$P25$ Raman}
\end{figure}
