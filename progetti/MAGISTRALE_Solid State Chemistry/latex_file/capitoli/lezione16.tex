\section{27/11/24}

\subsection{[$\checkmark$] LABORATORIO giorno 1 (27/11/24)}

\subsubsection{Calcoli stechiometrici PREPARAZIONE 1}

\paragraph{Objective:} Prepare 55 mL of solution, which needs to be:
\begin{itemize}
    \item 10 wt \% in glacial acid
    \item 0.1 M of $Bi(NO_3)_3 \text{ } 5H_2O$
\end{itemize}  

\noindent Knowing that the density of acetic acid is 1 Kg/L we can calculate that we need \textbf{5.5 mL} to create 55 mL of solution to match the 10 wt \% . To do this we just add to the 5.5 mL
of acetic acid the remaining water.

\vspace{10pt}

\noindent To create the final solution we know that the MW of $Bi(NO_3)_3 \text{ } 5H_2O$ is 485.07 g/mol, so to match the 0.1 M solution (55 mL) we will need to add \textbf{2.667885 g} of powder, knowing that 0.1M equals to $0.1 M * 55 mL = 0.0055 mol$ we calculate the grams with $0.0055 mol * 485.07 g/mol$

\vspace{10pt}

\noindent We measured the grams of Bi-complex: $2.67 g$

\vspace{10pt}

\noindent To prepare the KBr solution we had to find how much (in grams) we need to add to water; we multiplied the MW of KBr with the final volume and the molarity required:$33*10^{-3} L * 0.1 mol/L * 119.002 g/mol \approx$ \textbf{0.3927 g}

\paragraph{Execution:} We add in a becker the 55 mL of solution and the poweder, then we start the mixing with the magnetic stir.

\subsubsection{Calcoli stechiometrici PREPARAZIONE 2}

Same procedure, but with different quantites (\textbf{30 ml of total solution at 0.03 M of Bi} and \textbf{20 mL of Kbr at 0.025 M}, meaning: \textbf{0.44 g of Bi} and \textbf{3 mL of acetic acid}). Before the adding of the $KBr$ we add the 1 g of P25. The KBr will be diluted ONLY in water.

\paragraph{Reaction of the $Bi$} The reason why we work in an acid ambient is because the $Bi$ tends to precipitate as $BiO_2$, the acid environment prevents this, leaving the $Bi$ as an ion in the solution. Whenever the $KBr$ is added, it dissolves, binding the $Br^-$ to the $Bi^{3+}$. Due to the strong force of the $Bi^{3+}$, an $O^{2-}$ will bind itself to the molecule. In this LAB we don't have the capability to avoid it. Once it precipitates, we can filter it, removing the water, acetic acid and the potassium nitrate. In the second experiment with the P25, the precipitate bound to the P25, which remained inert during the reaction. This is a great way to precipitate the $Bi$ on a substrate.

%%%%%%%%%%%%%%%%%%%%%%%%%

\begin{figure}[ht]
    \centering
	\includegraphics*[width=0.22\linewidth]{../images/IMG20241127151359.jpg}
    \caption{Powder of Bismuto}
\end{figure}

\begin{figure}[ht]
    \centering
	\includegraphics*[width=0.22\linewidth]{../images/IMG20241127152740.jpg}
    \caption{Powder of Kbr}
\end{figure}

\begin{figure}[ht]
    \centering
	\includegraphics*[width=0.22\linewidth]{../images/IMG20241127152932.jpg}
    \caption{5.5mL of acetic acid}
\end{figure}

\begin{figure}[ht]
    \centering
	\includegraphics*[width=0.22\linewidth]{../images/IMG20241127153255.jpg}
    \caption{55mL of solution}
\end{figure}

\begin{figure}[ht]
    \centering
	\includegraphics*[width=0.22\linewidth]{../images/IMG20241127153705.jpg}
    \caption{magnetic stiring of Bi + acetic acid}
\end{figure}

\begin{figure}[ht]
    \centering
	\includegraphics*[width=0.22\linewidth]{../images/IMG20241127154319.jpg}
    \caption{setup for drops}
\end{figure}

\begin{figure}[ht]
    \centering
	\includegraphics*[width=0.22\linewidth]{../images/IMG20241127155233.jpg}
    \caption{added first drops}
\end{figure}

\begin{figure}[ht]
    \centering
	\includegraphics*[width=0.22\linewidth]{../images/IMG20241127155631.jpg}
    \caption{Powder of Bismuto}
\end{figure}

\begin{figure}[ht]
    \centering
	\includegraphics*[width=0.22\linewidth]{../images/IMG20241127155910.jpg}
    \caption{New solution of acetic acid (Part 2)}
\end{figure}

\begin{figure}[ht]
    \centering
	\includegraphics*[width=0.22\linewidth]{../images/IMG20241127161103.jpg}
    \caption{P25 pesato}
\end{figure}

\begin{figure}[ht]
    \centering
	\includegraphics*[width=0.22\linewidth]{../images/IMG20241127161909.jpg}
    \caption{Acetic acid + P25}
\end{figure}

\begin{figure}[ht]
    \centering
	\includegraphics*[width=0.22\linewidth]{../images/IMG20241127162709.jpg}
    \caption{KBr weighted}
\end{figure}

\begin{figure}[ht]
    \centering
	\includegraphics*[width=0.22\linewidth]{../images/IMG20241127163549.jpg}
    \caption{Adding KBr to acetic acid + P25}
\end{figure}

\ce{H2 + O2 -> H2O}
\\
\ce{CO2 + H2O <=> H2CO3}
\\
\ce{Na+ + Cl- -> NaCl}
\\
\ce{N2 + 3 H2 ->[\text{Fe, 400°C}] 2 NH3}

% 15:13 Powder of Bismuto
% 15:27 Powder of Kbr
% 15:29 5.5mL of acetic acid
% 15:32 55mL of solution
% 15:37 magnetic stiring of Bi + acetic acid
% 15:43 setup for drops
% 15:52 added first drops
% 15:56 Powder of Bismuto
% 15:59 New solution of acetic acid (Part 2)
% 16:11 P25 pesato
% 16:19 Acetic acid + P25
% 16:27 KBr weighted
% 16:35 Adding KBr to acetic acid + P25
