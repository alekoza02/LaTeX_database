\section{16/10/24}

\subsection{[$\checkmark$] Group work}
coordination number vs ionic radius:
Alcaline have only one electron in the outer shell (s1) so how can they have different coordination numbers? For instance, sodium cation can be in interaction with different anion sizes! The key is the ratio between the two radius.
In general, we can observe a linear proportionality between coordination number ad  ionic radius, but with sodium and silver  and potassium, we see a certain curvature due to different
numbers of oxidation.

Charge map: it's a map of electronic density. Things that can observe:
\begin{itemize}
    \item Of course the anion is bigger then the cation. 
    \item Electronic transfer from anion to cation.
    \item Different shapes: the rigid spheres model is a first approximation that doesn't work if we go deeper inside the observations. The rigid spheres work for the general rule: in order to have coordination, ions must touch with each other.
    \item The shape of the smaller ion seems to try reaching for the center of the bigger ion. This aggressive behaviour is observed more with ions with charge +1.
\end{itemize}

\subsection{[$\checkmark$] Rules}

\paragraph{Initial claims: }

\begin{itemize}
    \item Ions are spherical [False].
    \item Ions may be regarder as two different parts (core concentrated, outer zone small charge density) [False].
    \item Assignment of radii is difficult [True].
\end{itemize}

\paragraph{More modern version: }

\begin{itemize}
    \item For s and p elements radii increases with the atomic number.
    \item In isoelectronic series cations decreases with increasing charge.
    \item Same ion with different oxidation stat the cationc radius decreases.
    \item If more coordination numbers allowed, the radius increases with it.
    \item Strange behaviour in lanthanids.
\end{itemize}

\paragraph{More rules for ionic structures}

\begin{itemize}
    \item Consider the spheres as elastic and polarizable spheres, not rigid.
    \item High simmetry, next nearer neighbour interactions are repulsive.
    \item There's lots of void: local electroneutrality prevails. This also means that vacancies usually go in pairs.
    \item Useless 1.
    \item Useless 2.
\end{itemize}

\paragraph{Pauling's Rules (1929)}

\begin{itemize}
    \item Around each cation, a coordination polyhedron of anions is formed. The distance of the the ions distance is the radius sums, and the coordination number is determined by the radius ratio.
    \item A structure is stable as long as each ion is surrended a number of ions of the opposite charge that can equilibrate the starting ion. (Electroneutrality and Stechiometry)
    \item Shared edges and faces of same charge polyedra destabilize the crystal structure.
    \item If different cations present, the one with high valency and small coordination number tend not to share polyhedral elements (no shared edges or faces)
    \item There's a small number of different kinds of elements.
\end{itemize}

\subsection{[$\checkmark$] Exercise}

Consider a spinel as a ionic system, the electroneutrality is preserved, so we can calculate the Electrostatic Bond Strenght (EBS):

\begin{itemize}
    \item EBS = Cation charge / number of Anions
    \item $\sum$ m/n = Anion charge
\end{itemize}

Solution:

\begin{itemize}
    \item $M^{m+} + n X^{x-}$
    \item $Al^{3+} (Oh) \text{ EBS } = \frac{3}{6} = \frac{1}{2}$
    \item $Mg^{2+} (Td) \text{ EBS } = \frac{2}{4} = \frac{1}{2}$
    \item $\sum{\text{EBS}} = \frac{1}{2} + 3\frac{1}{2} = 2$
\end{itemize}

This is beacuse each Mg connects to 2 oxygen and each Al connects to 6 oxygen, considering that each oxygen connects to 1 Mg and to 2 Al. This brings us to this result.