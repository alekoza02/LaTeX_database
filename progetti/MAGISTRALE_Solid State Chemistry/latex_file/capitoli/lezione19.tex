\section{04/12/24}

\subsection{[$\checkmark$] LABORATORIO giorno 4 (4/12/24)}

\subsubsection{Theory XRD (Bragg - Brentano): }

We use a sample holder which has a transperent window made out of quartz that flattens the sample powder. 
This gives us the possibility to analize the sample using a dual motion of X-RAY emitter and detector, more precisely the both rotate of an angle $\theta$ from the horizon to the zenith.
The diffraction pattern will have on the X axis the 2$\theta$ value, and on the Y axis the intensity (counts) of the diffracted beam.

When comparing a sample with the reference, if the angle position of the most intense peak of the reference corresponds with the one of the most intense peak of the sample, means that most probably there is a good correspondance.

\subsubsection{Observation: }

When we compare the diffraction patterns of $TiO_2$, $BiOBr @ TiO_2$ and $BiOBr$ we notice that between the 25-26° there's a possible overlap in the $BiOBr @ TiO_2$ generated by the peaks of $BiOBr$ and $TiO_2$. This is purely an observation of the plots, without any refinement procedure.

\begin{figure}[ht]
    \centering
	\includegraphics*[width=0.7\linewidth]{../Group_5/grafici/XRD_overlap_white.png}
    \caption{Peaks overlapping}
\end{figure}