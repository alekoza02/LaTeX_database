\section{28/11/24}

\subsection{[$\checkmark$] LABORATORIO giorno 2 (28/11/24)}

\paragraph{Filtering Solution 1:}

We created the following setup and washed the product 5 times. 
After that we checked the pH of the waste solution, which was equal to 1-2.
We than washed again the product with only deionized water to check for remaining of acetic acid in the product.
We checked the pH to ensure there was none (pH neutrality) and we confirmed it by measured a pH $\sim$ 6-7.

\vspace{10pt}

\begin{minipage}{0.48\textwidth}
    \centering
    \includegraphics[width=0.6\linewidth]{../Group_5/photo_report/IMG20241128150335.jpg}
    \captionof{figure}{Setup for filtration}
\end{minipage}%
\begin{minipage}{0.48\textwidth}
    \centering
    \includegraphics[width=0.6\linewidth]{../Group_5/photo_report/IMG20241128142141.jpg}
    \captionof{figure}{Filtration process}
\end{minipage}

\vspace{10pt}

\begin{minipage}{0.32\textwidth}
    \centering
    \includegraphics*[width=0.9\linewidth]{../Group_5/photo_report/IMG20241128143021.jpg}
    \captionof{figure}{pH of waste-solution}
\end{minipage}%
\begin{minipage}{0.32\textwidth}
    \centering
    \includegraphics*[width=0.9\linewidth]{../Group_5/photo_report/IMG20241128143352_ruotata.jpg}
    \captionof{figure}{pH of filtered water}
\end{minipage}%
\begin{minipage}{0.32\textwidth}
    \centering
    \includegraphics*[width=0.9\linewidth]{../Group_5/photo_report/IMG20241128143600.jpg}
    \captionof{figure}{pH of deionized water}
\end{minipage}

\newpage

\paragraph{Results: } We obtained the product that we put in a glass container with the signature "4".

\vspace{10pt}

\begin{minipage}{0.48\textwidth}
    \centering
    \includegraphics[width=0.6\linewidth]{../Group_5/photo_report/IMG20241128142353.jpg}
    \captionof{figure}{Product after filtration}
\end{minipage}%
\begin{minipage}{0.48\textwidth}
    \centering
    \includegraphics[width=0.6\linewidth]{../Group_5/photo_report/IMG20241128143922.jpg}
    \captionof{figure}{Final product}
\end{minipage}

\paragraph{Filtering Solution 2:}

Same procedure, we had some problems with the filtration since the first time some product leaked in the waste water and the second time the setup exploded. The titania also tends to stick to the filter, blocking the filtering process. We had to filter and refilter the product, because once we start cleaning it, the eccess water damages the filtration process.

\vspace{10pt}

\begin{minipage}{0.48\textwidth}
    \centering
    \includegraphics[width=0.6\linewidth]{../Group_5/photo_report/IMG20241128160032.jpg}
    \captionof{figure}{Waste water, notice the not complete transparency}
\end{minipage}%
\begin{minipage}{0.48\textwidth}
    \centering
    \includegraphics[width=0.6\linewidth]{../Group_5/photo_report/IMG20241128160326.jpg}
    \captionof{figure}{pH of the filtering waste water}
\end{minipage}

\paragraph{Results: } We obtained the product that we put in a glass container with the signature "TiO2". It's worth mentioning that the waste solution we obtained is not crystal clear, but has a white gradient. This is most probably caused by the passage of nanoparticles of titania through the filter.

\vspace{10pt}

\begin{minipage}{0.48\textwidth}
    \centering
    \includegraphics[width=0.6\linewidth]{../Group_5/photo_report/IMG20241128160027.jpg}
    \captionof{figure}{Product with broken filter (water still present)}
\end{minipage}%
\begin{minipage}{0.48\textwidth}
    \centering
    \includegraphics[width=0.6\linewidth]{../Group_5/photo_report/IMG20241128160558.jpg}
    \captionof{figure}{Final product}
\end{minipage}
