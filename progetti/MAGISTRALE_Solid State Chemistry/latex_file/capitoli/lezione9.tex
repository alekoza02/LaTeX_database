\section{28/10/24}

\subsection{[$\checkmark$] Metal Oxides}

They can be classified based on the valence band:

\begin{itemize}
    \item 2p orbitals of oxygen filled (transfer from S of the metal $\rightarrow$ from the 2° column). It needs LOTS of energy, and it depends on the width of the band gap.
    \item Defects of other materials that can generate SOME new values inside the energy gap. P-TYPE (close to valence band). This type can also have some levels just beneath the conduction band. They can generate due to different coordination number due to defects. 
    \item N-TYPE semiconductor
    \item Conductor
\end{itemize}

\textit{Immagine sulla classificazione degli ossidi}

\noindent It's also worth mentioning that not only the band gap is important, but also the band width. If they are very compact, it will result in more strict energy levels for jumps.

\paragraph{New model:} We can no longer consider the atoms as individual rigid spheres, now certain orbitals can penetrate one another.

\paragraph{Magentism:} In certain metals, due to electron transfer, we can have a cartain number of electrons oriented in such way that the total magnetic number is $\neq 0$, this means that is exibits magnetic behavior. This leads to contraddiction with the band theory, where the band gap is inside of the same orbital, resulting in magnetic insulator.