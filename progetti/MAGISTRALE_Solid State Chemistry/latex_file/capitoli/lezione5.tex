\section{14/10/24}

\subsection{[$\checkmark$] Microwaves}

\paragraph{Advantages}
\begin{itemize}
    \item Rapid
    \item Cheap
    \item Clean
    \item Suitable for many materials
\end{itemize}

\paragraph{Group of materials}

\begin{itemize}
    \item Reflectant (conductors)
    \item Transparent (everything that can't absorb)
    \item Absorber (water, but depends on the $\nu$)
\end{itemize}

\paragraph{Heating:} The way microwaves heat is the opposite of the traditional thermal exchange, this is because microwave heating starts from the center. The heating can happen in two different ways:

\begin{itemize}
    \item Joule effect: movement of charged particles (conductors only)
    \item Dipol moment: particles that interact with the electric field, the rotation it generates to alingn creates heat. In solids it happens too, but the oscillation must be slower. This is why we use microwaves and not for example ultraviolet. 
\end{itemize}

\paragraph{Energy of the microwave:} It's very little, since they are in the range of $10^{-6} - 10^{-4} eV$, while even a C-H bond is $4.51 eV$. Nonetheless it's enough to heat these compounds.

\paragraph{Usages:} $SiC$, and Perovskites

\subsection{[$\checkmark$] Sintering process}

\paragraph{What is it?:} it's the result of thermal treatmant beyond the fusion temperature, it increases the density and diminuish the porosity. This happens because it's a spontaneous process that lead to the lowering of the system free energy (surface energy) that leads to a lower surface area. It's considered a transportation mechanism.

\paragraph{How does it occour?} It happens by generating a neck between particles that keep growing. But sometimes some porosity remains enclosed. In the final phase, the particles start growing into the inside, pushing away any remaining porosity. The particles get closer initially because of the difference of free energy. The best shape ia 6 side polyedron.

\paragraph{Shape matters:} Based on the shape we can predict the crystallographic orientation, concavity and if a small particle gets "eaten" by a bigger one.

\paragraph{Pores:} If they are on the grain boundary can be easily removed, if the boundary is moving fast, cannot be removed because it gets trapped. Slower reaction (temperature) more dense material. This is why the grains cannot grow that much in general, because to do this, porosity are much more propense to get trapped.

\subsection{[$\checkmark$] Structure}

\paragraph{postulate:} The most probable structures will be those in which the most economical use of space is made. The nature itself tries to lower it as much as possible. 


Influences:

\begin{itemize}
    \item the bond type will be based o the element nature: 
    \begin{itemize}
        \item metal $\rightarrow$ non directional
        \item covalent crystals $\rightarrow$ directional bonding
        \item ionic crystals $\rightarrow$ non directional between polar elements.
    \end{itemize}
    \item Electronic configuration
    \item Atomic dimensions
\end{itemize}
%
For example, between $SrO$ $BaO$, $HgO$ is not ionic, even if based on the electronegativity could lead to ionic state. This is because of the electronic state, that leads to a hybridization, and not an electronic exchange.

\vspace*{10pt}

\noindent Another example is based on the size: looking at the ``alogenuri" we see that the bond type goes from strongly ionic to covalent.