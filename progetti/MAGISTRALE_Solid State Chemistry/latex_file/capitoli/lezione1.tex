\section{30/9/24}

\subsection{[$\checkmark$] Introduction to Solid State Chemistry}

\begin{itemize}
	\item Introduction
	\item Crystalline Solids
	\item Solid state synthesis
	\item Bond models: Electronic structure of Solids
	\item Structure and properties of metal oxides
	\item Low dimensional solids
	\item Defects in Crystals
	\item Basis for EPR spectroscopy
\end{itemize}

It's important to notice that shortages of materials are important across all types of environments, ranging from electronics to agricolture (Fosforo P). It's becoming more cost-effective to recycle than to mine new materials.

Hot take: wind turbines use 300 kg of $Nd$. It's VERY expensive, consider it.

\subsection{[$\checkmark$] Presentation crystalline structure}

\paragraph{ZincBlenda:} ZincBlenda is a type of crystalline structure made out of $Zn$ and $S$ based on a "Corp Centered Package" (CCP). The lattice is a "Face Centered Cubic" (FCC). The unit cell can be subdivided in 8 octants, each of them forms a tetrahedral (4 verteces made of Zn and 1 atom of S in the middle). This means that the coordination is 4:4. Many combinations of elements can be structured in this lattice, the most famous are:
\begin{itemize}
	\item $GaAs$ (ionic bond)
	\item Diamond (covalent bond)
\end{itemize}

Whenever the bond type is ionic, it can be easily dissolved in water.

\hrulefill

\paragraph{GENERIC: Fluorite / AntiFluorite} It's a CCP of cations with anions in all tetrahedral holes. The lattice is an FCC, while the unit cell is made out of $4 CaF_2$. The coordination for cations is 8 (cubic) and for anions is 4 (tetrahedral). The name difference is based on which type of ion forms which lattice. The difference between Fluorite and AntiFluorite is based on which type of anion is inside the tetrahedral holes.

\begin{itemize}
	\item Fluorite: anions in tetrahedral hole
	\item AntiFluorite: cations in tetrahedral hole
\end{itemize}

Please notice that the AntiFluorite is not a true CP, since the anions don't touch.