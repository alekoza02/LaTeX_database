\section{4/10/24}

\paragraph{Spinels}

(Last part of the previous lession) General form: $[A]^{td}[B]^{oh}O_4$, here once again big cations small charge and small cations big charge.

\subsection{RECAP OF FIRST CHAPTER}

\begin{itemize}
    \item Describe a structure when you observe it (evaluate coordination number of each ion in the structure, (rutile $\rightarrow$ small cations, big anion (usually the structure is made by anions (not this case)). Oxygen tri-coordinated in a plane, titanium (octa?)))
    \item Saying if a compound is a bi-tri-tetra compound.
\end{itemize}

\subsection{[$\checkmark$] GROUP: Inorganic materials synthesizing methods}

\begin{itemize}
    \item Sputtering of ions.
    \item Fusion and crystallization of metals.
    \item Crystal growth in saturated solution.
    \item Nuclear fusion / fission.
    \item SolGel for production of nanoparticles.
    \item RedOx to generate electricity.
\end{itemize}

\subsection{[$\checkmark$] Synthesis of solid state inorganic material}

There are some parameters that generally guide the reaction:

\paragraph{Area of contact between reacting solids: }

\begin{itemize}
    \item Reactants need to have a large surface area.
    \item Pelletize to favour close contact between crystallites.
\end{itemize}

\paragraph{Rate of diffusion: }

\begin{itemize}
    \item High temperature.
    \item Introducing defects that favour the reaction's start.
\end{itemize}

\paragraph{Rate of nucleation: }

\begin{itemize}
    \item Change of the conditions at which atoms start bounding together.
\end{itemize}

There are also different materials we can synthetize:
\begin{itemize}
    \item Single crystals
    \item Polycrystalline
    \item Thin films
    \item Glass fibers
\end{itemize}

\paragraph{Single crystals: }During the synthesis, based on the state of the reactants, you can have homogenous and eterogenous states.

\subsection{[$\checkmark$] Processes}

\paragraph{Bridgman - Stockbarger process: }

Production of $Si$, which is based on using a seed (small crystal of pure material). This will result into the cristallization of the melted material that grows in the same direction of the original seed.

\paragraph{Czochralski process: }

Is a very similar method of the Bridgman process, but the seed holder is rotating. It also pulls up to colder regions, accellerating the growth process.

\paragraph{Verneuil: }

Used for production of oxydes, like for example $Al_2O_3$, the problem that usually the crucible is made out of Alumina :). The way it works is to make it almost drop on the flame and then make it fall on the growing crystal (like stalactite).

\paragraph{Crystallization in oversaturated solutions (tipologies): }

For monocrystals we can use Hydrothermal synthesis: Used for zeolites, uses high pressure and temperature.

\vspace{15pt}

\noindent For polycrystals the simplest way is to mix powders and heat. The downside is that we need a very high temperature.

\vspace{15pt}

\noindent We can combine the with direct combination reactions, which create a completely new compound that is nothing like the starting material, we use a phase diagram to understand what we will obtain. This happens because of ions diffusion, and they will need to be balanced (Kirkendall effect).

\subsection{[$\checkmark$] Techniques}

\paragraph{CVD} The usage of Chemical Vapour Deposition (CVD) which deposits vapours material to create: amorphous materials, crystalline solids, epitaxial growth, single crystals. It is based on thermal decomposition of a gas or chemical reaction between 2 gases. We can also be more precise in the reaction that happens using hydration, redox and substitution.

The common thing is the usage of $Cl$ which creates volatile compounds, which then will be easier to deposit. Since it's toxic, all the reactors are equipped with an exhaust to eliminate the $Cl$. Precursors need to have these properties: Volatile, Stable, Reactive and Safe.

\vspace{15pt}

\noindent Changing the temperature, substrate and oversaturation we can get differnt type of morphology. Studying this behaviour gives us another degree of freedom to generate different materials.

\paragraph{Pro: }
\begin{itemize}
    \item Very high adhesion
    \item High density material
\end{itemize}
\paragraph{Cons: }
\begin{itemize}
    \item Slow deposition
\end{itemize}
\paragraph{Applications: }
\begin{itemize}
    \item Superconductors
    \item Diffusion barriers
    \item Thermal barriers
    \item Optical
\end{itemize}