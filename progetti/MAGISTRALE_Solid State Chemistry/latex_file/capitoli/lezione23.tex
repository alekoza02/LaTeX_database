\section{13/12/24}

\subsection{[$\checkmark$] LABORATORIO giorno 8 (13/12/24)}

\subsubsection{Photocatalysis: }

\paragraph{Day 1:} 
Preparation of the suspension (BiOBr)
\begin{itemize}
    \item Weight 10 mg (real: 10.4 mg) of BiOBr powder washing the polymeric boat, put it in a 100 ml flask with a bit of water and put in the ultrasounds water, so that we have a suspension. Then we added phenol( 1 ml ) and added water until the volume line. Then we put 15 ml of the suspension in the irradiation cell with magnet. We took 0.5 ml with a siring and an attached filter in a little flask. This is the 0 minutes sample. Then we irradiated the suspension (300-400 nm, 20 $W/m^2$) for times: 2.5 min (3 actually), 5, 10, 15, 30, 1 h and 24 hours. Every time we irradiated, we sampled the suspension and we took an HPLC analysis in order to see residues of phenol ad so the concentration of our compound of interest.
    
    \item So calibration curve for the concentration of phenol: area vs concentration of 4 standards so we find the concentration of our 8 samples;
    \item Exponential curve: $e^{KT}$ (K is the rate constant) for the degradation of phenol. so we have concentration of each sample vs time of irradiation.
\end{itemize}

\paragraph{Day 2:} 
we did the same but with BiOBr@TiO2. Weight 10 mg (real: 10.3 mg) of BiOBr@TiO2 powder using the doble weight technique, so we first weighted the sample inside the boat -> 11.4 mg; then we weighted the boat alone -> 0.11 mg. Then we took the difference. 

We have to put eeuueeeum no questo lo teniamo, we will analize it later, first thing we need to report everything, but i think it's already done, so how much ml in every sample, eeeeee the wavelenght, the èpowerof the irraditaion, then he told us, SI, sisi, il range, we  use the range, poi, the second thing he told, oh my gosh AH SI with the data of our sample we build a curvbe using an exponential which is $e^{KT}$ i think it's $-KT$, because it's an exponential that goes to zero eeeeee, yes the calibration curve is for the calculation of theeee OH MIO DIO, but it's a different thing. You are stupid as fuck, alright let's talk about the calibration curve in order to find the concentration: so we have data of 4 standards, adesso ti dico cosa scrivere, they give us: so we use a linear interpolation of the standards data and we find the unknown concentration of the sample. 