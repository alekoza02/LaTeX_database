\section{2/10/24}

\subsection{[$\checkmark$] Diffraction}

Best instrument to differentiate between for example $TiO_2$ (\textit{Anatase, Brookite, Rutile})

\subsection{[$\checkmark$] Ternary compounds}

\subsubsection{Perovskite, $ABO_3$}

\begin{figure}[ht]
    \centering
    \includegraphics[width=.9\textwidth]{../images/perovskite.png}
    \caption{Crystalline structure of Perovskite}
    \label{fig:perovskite}
\end{figure}

\begin{itemize}
    \item A is a big cation (for example alkalyne and alkalyne terr.)
    \item B is a small cation with big charge (for example from $TiO_2 \rightarrow Ti^{4+}$)
    \item C is Oxygen 
\end{itemize}
%
The mother of all Perovskite is $CaTiO_3$, where the big cation can be $Ca$ or $Sr$, with maximum number of coordination: 12.

Another type of perovskite for example is $La_2CuO_4$ which is a high temperature Superconductor. In this case it's more a stacked structure (sandwich $ABA^{-1}$). It's important to remember that a stacked perovskite exists for the future.

\subsubsection{New-Perovskite, $Pb$}

This is a type of Perovskite based on $Pb^{2+}$ as a small cation, a big organic cation $CH_3NH_3^+$ and instead of the Oxygen we have an alogen $X^-$ ($I$, $Br$, $Cl$). It not very good, since $Pb$ is dangerous, we should be avoiding it. In addition, since they are very sensible towards water, it's very difficult to create solar panles based on them, beacause it can dissolve itself in the nature.

\subsubsection{Perovskite Superconductors}

An example of them is the $YBa_2Cu_3O_7$

\subsubsection{Spinels: $[A]^{th}[B_2]^{oh}O_4$ and inverse-Spinels: $[B]^{td} [A,B]^{oh} O_4$ }

The most important is the $MgAl_2O_4$ where again A and B are cations, but this time B is interstial.

In the inverse case, we have an example of $Fe_3O_4$ which can have 2 different types of configuration inside the same structure $\rightarrow [Fe^{3+}]^{td}[Fe^{2+}, Fe^{3+}]^{oh}O_4$

\subsubsection{Crystalline Field Theory}

The theory behind transition metal that transfer electrons to the atoms around them, based on the electronic configuration and form of the orbitals we can obtain different shapes and consequentely different energies. This theory can be applied to solids also, as if the crystal serves as a solvent.

\begin{figure}[ht]
    \centering
    \includegraphics[width=.9\textwidth]{../images/CFT.png}
    \caption{Scheme of Crystalline Field Theory}
    \label{fig:CFT}
\end{figure}